\section{Witness Encryption}

There is a strong connection between $\we$ and computational secret sharing for $\NP$~\cite{STOC:GGSW13}.

\heading{Related Works}
\begin{itemize}
    \item \cite{STOC:GGSW13} They have introduced the first witness encryption scheme.
    \begin{itemize}
        \item First definition of witness encryption.
        \item Construction of $\pke$, $\ibe$, $\abe$, adaptively-secure $\ibe$ schemes from $\we$.
        \item Construction of $\we$ from the multi-linear map.
    \end{itemize}
\end{itemize}

\subsection{Definition}

We define the witness encryption scheme here. It was firstly proposed in~\cite{STOC:GGSW13} as a concept for building attribute-based encryption schemes ($\abe$).

\begin{definition}
    A witness encryption scheme for an $\NP$ relation $\Rel \subseteq \bin^{n(\secpar)} \times \bin^{m(\secpar)}$ with a corresponding $\NP$ language $\Lang = \{\statmnt : \exists \wit. (\statmnt,\wit) \in \Rel\}$ consists of three $\ppt$ algorithms $\we = (\KGen, \Enc, \Dec)$ with the following syntax.
    \begin{trivlist}
        \item $\KGen(\secparam, \Rel) \to \ek:$
        \item $\Enc(\ek, \statmnt, \msg) \to \ct:$
        \item $\Dec(\ek, \ct, \wit) \to \msg:$
    \end{trivlist}
    We require the following properties:
    \begin{trivlist}
        \item \Prop{Correctness:} For all $\secpar \in \NN$, all $(\statmnt, \wit) \in \Rel$, and all $\msg \in \MsgSp$, we have:
        \begin{align*}
            \condprob{\Dec(\ek, \Enc(\ek, \statmnt, \msg)) = \msg}{\ek \getsr \KGen(\secparam, \Rel)} &= 1- \negl.
        \end{align*}
        \item \Prop{Security:} For all two-stages $\ppt$ adversary $\adv = (\adv_1, \adv_2)$, we have:
        \begin{align*}
            \left|\condprob{b' = b \land \statmnt \notin \Lang}{\begin{array}{l}
                \ek \getsr \KGen(\secparam, \Rel)\\
                (\msg_0, \msg_1, \statmnt, \state) \getsr \adv_1(\ek)\\
                b \getsr \bin\\
                \ct_b \getsr \Enc(\ek, \statmnt, \msg_b)\\
                b' \getsr \adv_2(\state, \ct_b)
            \end{array}}- \frac{1}{2}\right| & = \negl.
        \end{align*}
    \end{trivlist}
\end{definition}

\subsection{Applications}

\subsubsection{$\pke$ from $\we$}

\subsubsection{$\ibe$ from $\we$}

We present here the $\ibe$ construction from $\we$ schemes. We also need an unique signature scheme $\SIG = (\KGen, \Sig, \Ver)$ together with a Goldreich-Levin~\cite{STOC:GolLev89} hardcore bit $\hb$ (\cref{sec:hardcore-bit}).

We can introduce the notion of message-recoverable signature scheme, which means given a signature $\sigma$, it is easy to get a message $\msg$, such that $\SIG.\Ver(\vk, \msg, \sigma) =1$. Note that such scheme can be easily constructed from any signature scheme by appending the message directly after the signature. 

Note that given an unique signature $\sigma$ with message recoverability, the message recover algorithm $\SIG.\MsgRec(\mpk, \sigma) \to \hash(\msg)$ of signature scheme is actually an one-way function. Therefore, we can define the hardcore bit function $\hb$ of the above one-way function as $\hb_{\vec{r}}(\sigma)\gets \vec{r} \cdot \sigma$.

For the witness encryption scheme we define the following two $\NP$ relations.
\begin{align*}
    ((\id, \vec{r}), \sigma) \in \Rel_0 & \Leftrightarrow \SIG.\Ver(\mpk, \id,  \sigma) = 1 \land \hb_{\vec{r}}(\sigma) = 0 \\
    ((\id, \vec{r}), \sigma) \in \Rel_1 & \Leftrightarrow \SIG.\Ver(\mpk, \id,  \sigma) = 1 \land \hb_{\vec{r}}(\sigma) = 1
\end{align*}

\begin{figure}[ht!]%
	\centering%
	\nicoresetlinenr
	\fbox{%
		\begin{minipage}[t]{.45\textwidth}
			\calg{$\Setup(\secparam)$}%
			\begin{nicodemus}
				\item $(\vk, \sk) \getsr \SIG.\KGen(\secparam)$
				\item $\mpk \gets \vk;\; \msk \gets \sk$
				\item \pcreturn $(\mpk, \msk)$
			\end{nicodemus}
		\medskip
            \calg{$\Enc(\mpk, \id, \msg)$}
            \begin{nicodemus}
                \item $\SigDim \gets \log_2|\SIG.\Sigma|$
                \item $\vec{r} \gets \bin^\SigDim$
                \item $\pk_0 \getsr \we.\KGen(\secparam, \Rel_0)$
                \item $\pk_1 \getsr \we.\KGen(\secparam, \Rel_1)$
                \item $\ct_0 \getsr \we.\Enc(\pk_0, (\id, \vec{r}), \msg)$
                \item $\ct_1 \getsr \we.\Enc(\pk_1, (\id, \vec{r}), \msg)$
                \item \pcreturn $\ct = (\id, \vec{r}, \ct_0, \ct_1)$
            \end{nicodemus}
		\end{minipage}
		\qquad
		\begin{minipage}[t]{.45\textwidth}
			\calg{$\KGen(\msk, \id)$} 
			\begin{nicodemus}
				\item $\sigma \getsr \SIG.\Sig(\sk, \id)$
			\end{nicodemus}
        \medskip
            \calg{$\Dec(\sk_\id, \ct)$}
            \begin{nicodemus}
                \item \pcparse $(\id, \vec{r}, \ct_0, \ct_1) =: \ct$
                \item \pcif $\sigma^\trans \cdot \vec{r} = 0$ \pcthen
                \item \tab $\msg \gets \we.\Dec(\pk_0, \sigma, \ct_0)$
                \item \pcelse
                \item \tab $\msg \gets \we.\Dec(\pk_1, \sigma, \ct_1)$
                \item \pcreturn $\msg$
            \end{nicodemus}
		\end{minipage}
	}%
	\caption{Construction of $\ibe$ scheme from $\we$.}
	\label{const:we2ibe}
\end{figure}

\Prop{Proof intuitions:} We try to modify the challenge ciphertext such that the $\ct_0$ encrypts $\msg_b$ and $\ct_1$ encrypts $\msg_{1-b}$. If the adversary answers $0$, we will answer that the hardcore bit is $0$, we will answer $1$ otherwise. Note that since our objective if to have the winning probability is non-negligibly far from $\frac{1}{2}$, it does not matter if we guess $b$ to $1-b$.

{\color{red} In the original paper of~\cite{STOC:GGSW13}, the proof is actually wrong here, without knowing the secret signing key, the simulator cannot really determine whether we should change $\ct_0$ or $\ct_1$. On the other hand, if the simulator knows the signing key then the function from $\sigma$ to $\msg$ is no longer an one-way function.}

\subsubsection{$\abe$ from $\we$}

\subsubsection{Adaptively-secure $\ibe$ from $\we$}