\subsection{Litterature}
\begin{itemize}
    \item \cite{C:BosTakTib22}: DOTT: First Lattice-based One-Round Online multi-signature scheme
    \item \cite{C:BosTakTib22}: MuSig-L: Lattice-based multi-signature
    \item \cite{C:Chen23}: DualMS: lattice-based dual-mode multi-signature 
    \item \cite{EC:PanWag23}: First Multi-Signature without forking lemma
    \item \cite{PKC:PanWag22}: First Tightly-secure signature in the multi-user setting
    \begin{enumerate}
        \item Tight security in the multi-user setting requires random-self reducibility, which implies the tight worst-case to average-case reduction. Therefore, only LWE assumption is the potential candidate.
    \end{enumerate}
\end{itemize}

\subsection{Definition}

\begin{definition}[Multi-Signature]
    A (two-round) multi-signature scheme consists of four $\ppt$ algorithms $\MSig = (\Setup, \KGen, \Sig, \Ver)$ with the following syntax:
    \begin{itemize}
        \item $\Setup(\secparam) \to \pp$ takes the security parameter $\secparam$ as input and returns public parameter $\pp$. We assume that $\pp$ implicitly defines the space of public keys, secret keys, messages, and signatures, respectively. All algorithms related to $\MSig$ take at least implicitly $\pp$ as input.
        \item $\KGen(\pp) \to (\pk, \sk)$ takes the public parameter $\pp$ as input, and returns a public key $\pk$, and a secret key $\sk$.
        \item $\Sig = (\Sig_0, \Sig_1, \Sig_2)$ is a three staged algorithms:
        \begin{itemize}
            \item $\Sig_0(\Parties, \sk, \msg) \to (\pmsg_1, \st_1)$ takes a list of public keys $\Parties = (\pk_1, \ldots, \pk_{\NumP})$, a secret key $\sk$, and a message $\msg$ as input, and returns a protocol message $\pmsg_1$, and a state $\st_1$.
            \item $\Sig_1(\st_1, \MSet_1) \to (\pmsg_2, \st_2)$ takes as input a state $\st_1$ and a tuple of protocol messages messages $\MSet_1 = (\pmsg_{1,1}, \ldots, \pmsg_{1, \NumP})$, and returns a protocol message $\pmsg_2$ and a state $\st_2$.
            \item $\Sig_2(\st_2, \MSet_2) \to \sigma$ takes as input a state $\st_2$ and a tuple of protocol messages messages $\MSet_2 = (\pmsg_{2,1}, \ldots, \pmsg_{2, \NumP})$, and returns a signature $\sigma$.
        \end{itemize}
        \item $\Ver(\Parties, \msg, \sigma) \to b$ takes a set of public keys $\Parties = (\pk_1, \ldots, \pk_{\NumP})$, a message $\msg$, and a signature $\sigma$ as input, and returns a bit $b \in \{0,1\}$.
    \end{itemize}
    We require the following properties:
    \begin{itemize}
        \item \heading{Completeness:} $\MSig$ is complete if for all $\pp \getsr \Setup(\secparam)$, all $\NumP \in \poly$, all $(\pk_j, sk_j)\getsr \KGen(\pp)$ for $j \in [\NumP]$, all $\PkSet = (\pk_1, \ldots, \pk_\NumP)$, all $\SkSet = (\sk_1, \ldots, \sk_\NumP)$, and all messages $\msg$, we have
        \begin{align*}
            \condprob{\Ver(\PkSet, \msg, \sigma) = 1}{\sigma \getsr \MSig.\Exec(\PkSet, \SkSet, \msg)} = 1,
        \end{align*}
        where the algorithm $\MSig.\Exec$ is defined in \cref{fig:def-complet-multi-sig}.
        \begin{figure}[h!]%
	   \centering%
	   \nicoresetlinenr
	   \fbox{%
		\begin{minipage}[t]{.45\textwidth}%
			\underline{\textbf{Alg} $\Exec(\PkSet, \SkSet, \msg):$}
			\begin{nicodemus}
				\item \pcparse $\PkSet  =: (\pk_1, \ldots, \pk_\NumP)$
                    \item \pcparse $\SkSet =:(\sk_1, \ldots, \sk_\NumP)$
                    \item \pcfor $i \in [\NumP] : (\pmsg_{1,i}, \st_{1,i}) \getsr \Sig_0(\PkSet, \sk, \msg)$
                    \item $\MSet_1 \gets (\pmsg_{1,1}, \ldots, \pmsg_{1, \NumP})$
                    \item \pcfor $i \in [\NumP] : (\pmsg_{2,i}, \st_{2,i}) \getsr \Sig_1(\st_{1,i}, \MSet_1)$
                    \item $\MSet_2 \gets (\pmsg_{2,1}, \ldots, \pmsg_{2, \NumP})$
                    \item \pcfor $i \in [\NumP] : \sigma_i \getsr \Sig_2(\st_{1,i}, \MSet_2)$
                    \item \pcif $\exists i \neq j \in [\NumP]. \sigma_i \neq \sigma_j:$ \pcreturn $\bot$
                    \item \pcreturn $\sigma \gets \sigma_1$
			\end{nicodemus}
		\end{minipage}
	   }%
	   \caption{The execution algorithm $\Exec$ for a (two-round) multi-signature scheme $\MSig = (\Setup, \KGen, \Sig, \Ver)$, representing an honest execution of the signing protocol $\MSig$.}
	   \label{fig:def-complet-multi-sig}
        \end{figure}
        \item \heading{Key Aggregation:} A multi-signature scheme $\MSig = (\Setup, \KGen, \Sig, \Ver)$ is said to support key aggregation, if the algorithm $\Ver$ can be split into two deterministic polynomial time algorithms $(\Agg, \VerAgg)$ with the following syntax:
        \begin{itemize}
            \item $\Agg(\PkSet) \to \bar{\pk}$ takes a public key set $\PkSet = (\pk_1, \ldots, \pk_\NumP)$ as input, and returns an aggregated public key $\bar{\pk}$.
            \item $\VerAgg(\bar{\pk}, \msg, \sigma) \to b$ is deterministic, takes an aggregated key $\bar{\pk}$, a message $\msg$, and a signature $\sigma$ as input, and returns a bit $b \in \{0,1\}$.
        \end{itemize}
        More precisely, algorithm $\Ver(\PkSet, \msg, \sigma)$ can be written as $\VerAgg(\Agg(\PkSet), \msg, \sigma)$. Note that we require succinctness of aggregated public key. Namely, the size of $\bar{\pk}$ is independent of the number of users.
        
        \item \heading{$\mseufcma$:} Let us consider the security game $\mseufcmaG$ defined in~\cref{fig:def-mseufcma}. We say that $\MSig$ is $\mseufcma$ secure, if for all $\ppt$ adversary $\adv$, the following advantage is negligible:
        \begin{align*}
            \mathsf{Adv}_{\MSig, \adv}^{\mseufcma}(\secparam) & \gets \prob{\mseufcmaG(\secparam) \Rightarrow 1}.
        \end{align*}

        \begin{figure}[ht!]%
	\centering%
	\nicoresetlinenr
	\fbox{%
		\begin{minipage}[t]{.45\textwidth}%
			$\underline{\mseufcmaG(\secparam)}$
			\begin{nicodemus}
				\item $\pp \getsr \Setup(\secparam)$ 0
                    \item $(\pk, \sk) \getsr \KGen(\pp)$
                    \item $\Oracle{\Sig} = (\Oracle{\Sig}_0, \Oracle{\Sig}_1, \Oracle{\Sig}_2)$
                    \item $(\PkSet^\star, \msg^\star, \sigma^\star) \getsr \adv^{\Oracle{\Sig}}(\pp, \pk)$
                    \item \pcif $\pk \notin \PkSet^\star \lor (\PkSet^\star, \msg^\star) \in \List{}:$
                    \item \tab \pcreturn $0$
                    \item \pcreturn $\Ver(\PkSet^\star, \msg^\star, \sigma^\star)$
			\end{nicodemus}
		\medskip
			\noindent \underline{\textbf{Oracle} $\Oracle{\Sig}_0(\PkSet, \msg):$}
			\begin{nicodemus}
				\item \pcparse $(\pk_1, \ldots, \pk_\NumP) =: \PkSet$
                    \item \pccheck $\pk_1 = \pk$
                    \item $\List{} \gets \List{} \cup \{(\PkSet,\msg)\}$
                    \item $sid \gets sid+1;\; ctr[sid] \gets 1$
                    \item $(\pmsg_1, \st_1)\getsr \Sig_0(\PkSet, \sk, \msg)$
                    \item $(\pmsg_1[sid], \st_1[sid]) \gets (\pmsg_1, \st_1)$
                    \item \pcreturn $(\pmsg_1[sid], sid)$
			\end{nicodemus}
		\end{minipage}
		\qquad
		\begin{minipage}[t]{.45\textwidth}
			\underline{\textbf{Oracle} $\Oracle{\Sig}_1(sid, \MSet_1):$}
			\begin{nicodemus}
				\item \pccheck $ctr[sid]\neq 1$
                    \item \pcparse $(\pmsg_{1,1}, \ldots, \pmsg_{1, \NumP}) =:\MSet_1$
                    \item \pccheck $\pmsg_1[sid] =  \pmsg_{1,1}$
                    \item $ctr[sid] \gets ctr[sid]+1$
                    \item $(\pmsg_2, \st_2) \getsr \Sig_1(\st_1[sid], \MSet_1)$
                    \item $(\pmsg_2[sid], \st_2[sid]) \gets (\pmsg_2, \st_2)$
                    \item \pcreturn $\pmsg_2[sid]$
			\end{nicodemus}
                \medskip
                \underline{\textbf{Oracle} $\Oracle{\Sig}_2(sid, \MSet_2)$}
                \begin{nicodemus}
				\item \pccheck $ctr[sid] = 2$
                    \item \pcparse $(\pmsg_{2,1}, \ldots, \pmsg_{2, \NumP}) =:\MSet_2$
                    \item \pccheck $\pmsg_2[sid] =  \pmsg_{2,1}$
                    \item $ctr[sid] \gets ctr[sid]+1$
                    \item $\sigma \getsr \Sig_2(\st_2[sid], \MSet_2)$
                    \item \pcreturn $\sigma$
			\end{nicodemus}
		\end{minipage}
	}%
	\caption{The security game $\mseufcmaG$ for a (two-round) multi-siganture scheme $\MSig$ and an adversary $\adv$. Without losing the generality, we assume the adversary cannot access $\sk$ corresponding to $\pk$, the first public key $\pk_1$ of the set $\PkSet$.}
	\label{fig:def-mseufcma}
        \end{figure}
    \end{itemize}
\end{definition}

\subsection{Some New Ideas}

Here we list some of the ideas:

\subsubsection{Tightly-secure Multi-Signature from $\cdh$ assumption}
We give some insights here
\begin{itemize}
    \item Our starting point is~\cite{C:Chevallier-Mames05}, which is a tightly-secure signature scheme based on $\cdh$ assumption.
    \item The problem we encounter here is that, as with all other multi-signature schemes, it is hard to simulate the signature without knowing the secret key. The core reason is that the adversary may generate its first-round message  $\mathsf{R}_1$ after seeing all honest ones. The potential solution is to use the random oracle as in~\cite{CCS:BelNev06}. Compare to~\cite{EC:PanWag23}, the random oracle is used as an equivocable commitment scheme.
    \item We do not need the homomorphism as in~\cite{EC:PanWag23}, this is because we directly open the commitment at round $2$. 
\end{itemize}

This does not work since we can only have $3$-round $\cdh$ based protocol. We need $2$ rounds only to check that the first commitments are correct in the ROM model. \cite{EC:PanWag23} does not need this because they homomorphically compute the commitment and check after the homomorphism. Maybe $3$-round tightly secure multi-signature from $\cdh$ assumption is feasible. But not sure this is a valuable result.

\subsubsection{Simpler generic construction of multi-signature scheme}

\begin{itemize}
    \item We can notice that the main difficulty of the construction of 
\end{itemize}