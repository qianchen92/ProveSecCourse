\section{Lattice trapdoors}

\subsection{Trapdoor of third-generation fully homomorphic encryption}

This is the gadget trapdoors from bit decomposition and rebuilding techniques, firstly presented in~\cite{EC:MicPei12,C:GenSahWat13}. Note that the lattice trapdoor of third generation is actually an "All-But-One"/"Punctured" trapdoor function or a punctured trapdoor function.

\begin{definition}[Punctured Trapdoor] A $\sis$-based punctured trapdoor scheme consists of three $\ppt$ algorithms $(\KGen, \TrapGen, \Eval, \SampleD, \SamplePre)$ with the following syntax:
\begin{trivlist}
    \item $\KGen(\secparam) \to \ek:$
    \item $\TrapGen(\secparam, \Tag^\star) \to (\ek, \tk_{\Tag^\star}):$
    \item $\Eval(\ek, \Tag, x) \to y:$
    \item $\SampleD(\ek, \Tag) \to x:$
    \item $\SamplePre(\ek, \tk_{\Tag_\MsgDim}, \Tag, y) \to x \cup \bot :$
\end{trivlist}

We also require the following properties.

\begin{trivlist}
    \item 
\end{trivlist}
\end{definition}

We give the following lattice-based trapdoor construction~\cite{EC:MicPei12}.



Informally, the above trapdoor function has linear homomorphism on the right side. We can compute $\sum_{j \in \setfont{J}} \Eval(\ek, \Tag, x_j) k_j = \sum_{j} \Eval(\ek, \Tag, x_j\cdot k_j)$ with some smallness restriction on the values of $k_j$ and $|\setfont{J}|$.

\begin{definition}[Almost Liear-Homomorphism]
    
\end{definition}

