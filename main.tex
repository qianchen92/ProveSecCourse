% --- -----------------------------------------------------------------
% --- Please carefully check these flags.
% --- If you submit, set `submission' to 1, if you are required to submit in lncs style,
% --- set `llncs' to 1, if your submission has to be anonymous, set `anonymous' to 1.
% --- -----------------------------------------------------------------
\def\submission{0}				% 1 = deactivates: pagebackref, colour of links, authnotes
\def\llncs{0}					% Documentclass: 1 = llncs, 0 = book.
\def\llncsqedsymbol{1}			% Adds a QED symbol to all proof environments when llncs=1
\def\anonymous{0}				% 0 = non-anonymous, 1 = anonymous. No notes, authors, institutes
% ---------------------------------------------------------------------
\def\acknowledgments{0}			% 0 = off, 1 = on. Auto-0 for anonymous = 1.
\def\pagelimit{}				% Set to n for warnings on pages n' > n. Empty -> no warning
\def\papertype{0}				% 0 = A4, 1 = letter. Auto-1 for llncs = 1.
\def\fullpage{1}				% 0 = off, 1 = on. Auto-0 for llncs = 1.
% ---------------------------------------------------------------------
\def\LNCSpreview{0}				% Sets llncs=1, paper cropped to LNCS size.
\def\overflow{1}				% Show overfull lines
\def\showlabels{0}				% Show all defined labels and references.
\def\authnotes{1}				% 0 hides TODO, NEW, ALERT and authnotes.
								% Auto-0 for anonymous = 1.
\def\dieordollar{0}				% Modifies \getsr. Choose from {0,1,2}.
\def\choosebibstyle{alpha}		% Enter your bib style here. Wipe temp files after switch.
								% auto-splncs03 when llncs=1.
% ---------------------------------------------------------------------
\ifnum\LNCSpreview=1
	\def\llncs{1}
\fi

\ifnum\llncs=1
	\documentclass[runningheads,orivec]{llncs}
\else
	\ifnum\papertype=0
		\documentclass[a4paper,10pt]{book}
	\else
		\documentclass[letterpaper,10pt]{book}
	\fi

	\ifnum\fullpage=1
		\usepackage{fullpage}
	\fi
\fi

% --- -----------------------------------------------------------------
% --- Some options for last-minute space saving.
% --- -----------------------------------------------------------------
\def\stuffedtitlepage{0}		% CHEATING: NOT LNCS anymore. Removes spaces from the title page
\def\abbrevref{0}				% Abbreviate enviroment names when referencing
\def\allowbreaks{0}				% Allow LaTeX to pagebreak in displaymode

% --- -----------------------------------------------------------------
% --- Title, short Title and date go here.
% --- -----------------------------------------------------------------
\def\titletext{%				% Put your paper title here
	Cryptographic primitives
}
\def\runningtitle{%				% Only for llncs = 1. Used for LNCS runningtitle.
	Cryptographic primitives
}

\date{\today}					% Only for llncs = 0. Leave blank for no date.

% ---------------------------------------------------------------------
% --- Publication information (1-4 tied to SPRINGER)
% --- -----------------------------------------------------------------

								% Pubinfo turned off for LNCSpreview=1
\def\choosepubinfo{0}			% 0 : no publication info.
								% 1 : cameraready. Proceedings not yet published
								% 2 : cameraready. Proceedings already published
								% 3 : <  25% new content comp. to published version
								% 4 : >= 25% new content comp. to published version
								% 5 : individual text

\def\pubinfoYEAR{}				% year of publication for \pubinfoselect = 2, 3, 4
\def\pubinfoSUBMISSIONDATE{}	% date of submission to IACR for \pubinfoselect=2
\def\pubinfoDOI{}				% DOI of publication for \pubinfoselect = 2, 3, 4
\def\pubinfoBIBDATA{}			% bibliographic data for \pubinfoselect = 3, 4
\def\pubinfoCONFERENCE{}		% conference for \pubinfoselect=1

\def\pubinfoEPRINT{}			% EPRINT number of your publication
\def\pubinfoindividual{%
	place your individual publication text here
}

% ------------- Please do not touch this block of code ----------------
\makeatletter
\expandafter\newcommand{\createauthor}[5]{%
	\@namedef{#1name}{#2}%
	\@namedef{#1running}{#3}%
	\@namedef{#1institute}{#4}%
	\@namedef{#1thanks}{#5}%
}
\expandafter\newcommand{\createinstitute}[4]{%
	\@namedef{#1instname}{#2}%
	\@namedef{#1mail}{#3}%
	\@namedef{#1number}{#4}%
}
\makeatother
\newcounter{authorcount}
\newcommand{\newauthor}[4]{
	\stepcounter{authorcount}
	\createauthor{\theauthorcount}{#1}{#2}{#3}{#4}
}
\newcounter{institutecount}
\newcommand{\newinstitute}[3]{
	\stepcounter{institutecount}
	\createinstitute{\theinstitutecount}{#1}{#2}{#3}
}
% --- -----------------------------------------------------------------
% --- The authors go here. Please provide the information as
% ---
% --- \newauthor{First Last}{F. Last}{Institute number}{Footnote text}
% ---
% --- F. Last is used for LNCS runningauthors.
% --- Please always provide four (possibly empty) arguments.
% --- -----------------------------------------------------------------
\usepackage{orcidlink}
\newauthor{Chen~Qian~\orcidlink{0000-0003-4429-7267}}{C.~Qian}{1}{}
%\newauthor{Second~Author}{S.~Author}{221b}{}
%\newauthor{Third~Author}{T.~Author}{0815}{}

% --- -----------------------------------------------------------------
% --- The institues go here. Please provide the information as
% ---
% --- \newinstitute{Institute's name}{emails of authors}{Institute number}.
% ---
% \newinstitute{Shandong University}{chen.qian@sdu.edu.cn}{1}
% --- email is automatically set in teletype.
% --- Institute number only relevant for llncs=0.
% ---
% --- Please always provide three (possibly empty) arguments.
% --- Do not switch to math mode for the institute number/character.
% ---
% --- To suppress 'No institute given' when llncs = 1, use: \newinstitute{}{}{}.
% --- -----------------------------------------------------------------

%\newinstitute{ABC}{all@abc.org}{10}
%\newinstitute{DEF}{all@def.org}{11}
%\newinstitute{HGI}{all@hgi.org}{12}

% --- -----------------------------------------------------------------
% --- If you want to have a certain mail as contact mail, put it here.
% --- Emails will be clickable and link to mailto:\contactmail.
% --- -----------------------------------------------------------------

\def\contactmail{%
	mail here
}

% --- -----------------------------------------------------------------
% --- Abstract and Keywords go here
% --- -----------------------------------------------------------------

\def\abstracttext{%
	This text aims to keep track of all useful cryptographic tools I have encountered during the research. Some of the definition maybe useful to 
}

\def\keywords{%
	keywords here
}

% --- -----------------------------------------------------------------
% --- Add your acknowledgments here
% --- -----------------------------------------------------------------

\def\acknowledgmenttext{
	acknowledgments here
}

% ------------- Please do not touch this block of code ----------------
\usepackage{amsmath,amssymb,amsfonts}
\ifnum\llncs=0\usepackage{amsthm}\fi
%\usepackage[dvipsnames]{xcolor} % always before tikz
%\usepackage{tikz}
% ---------------------------------------------------------------------
\ifnum\submission=1
	%\usepackage[pdfpagelabels=true,linktocpage=true,colorlinks=true]{hyperref}
\else
	%\usepackage[pdfpagelabels=true,linktocpage=true,pagebackref,colorlinks=true]{hyperref}
	\ifnum\llncs=0
		\newcommand*{\backref}[1]{(Cited on page~#1.)}
	\fi
\fi % If submission=1: pagebackrefs are deactivated
	% If submission=0 and not using lncs `Cited on page...' is added to each entry in hte biblio

% ---------------------------------------------------------------------
\def\titleofpdf{\titletext}		%automatically embeds the title into pdf info. Part 1
\def\authorsofpdf{%				%automatically embeds the author names your pdf info. Part 1
\ifcsname 5name\endcsname
	\csname 1name\endcsname~et~al.
\else
	\ifcsname 1name\endcsname
		\csname 1name\endcsname
		\ifcsname 3name\endcsname
			,\ \csname 2name\endcsname
			\ifcsname 4name\endcsname
				,\ \csname 3name\endcsname\ and \csname 4name\endcsname
			\else%
				\ and\ \csname 3name\endcsname
			\fi
		\else
			\ifcsname 2name\endcsname
				\csname 2name\endcsname
			\fi
		\fi
	\fi
\fi
}

\ifnum\submission=1 %link colours are set to black when submission=1
	\hypersetup{
	pdftitle={\titleofpdf}, %embedding of title into pdf info. Part 2
	pdfauthor=\ifnum\anonymous=0{\authorsofpdf}\else{}\fi, %author pdf info embed. when not anonym.
	linkcolor=[rgb]{0,0,0}, %link colours are set to black.
	urlcolor=[rgb]{0,0,0},  %link colours are set to black.
	citecolor=[rgb]{0,0,0}  %link colours are set to black.
	}
\else
	\hypersetup{
	pdftitle={\titleofpdf},
	pdfauthor=\ifnum\anonymous=0{\authorsofpdf}\else{}\fi,
	linkcolor=[rgb]{0,0,0.5},
	urlcolor=[rgb]{0,0,0.5},
	citecolor=[rgb]{0,0.5,0}
	}
\fi
% ---------------------------------------------------------------------

%load cleveref always after hyperref
\ifnum\abbrevref=0
	\usepackage[capitalise,noabbrev]{cleveref}
\else
	\usepackage[capitalise]{cleveref}
\fi

\usepackage{nicodemus}
\usepackage[absolute]{textpos}	% used for warning messages and publication info
\usepackage{everypage}			% used for the page limit warning



\ifnum\LNCSpreview=1 %LNCS review reminder watermark
	\usepackage[paperwidth=152mm,paperheight=235mm,textwidth=122mm,textheight=193mm]{geometry}
	\AddEverypageHook{
		\begin{textblock}{1}[1,1](0.5,0.9)
			\centering\Large
			\textcolor{lightgray}{\textbf{LNCS Preview mode active.}}
		\end{textblock}
	}
	\def\choosepubinfo{0}
\fi

\ifnum\overflow=1
	\overfullrule=2mm
\fi

\ifnum\allowbreaks=1
	\allowdisplaybreaks
\fi

\ifnum\showlabels=1
	\usepackage{showkeys}
\fi

\ifnum\submission=1
	\def\authnotes{0}
\fi

\ifnum\anonymous=1
	\def\authnotes{0}
\fi

\ifx\pagelimit\empty %page limit warning magic
\else
	\newcounter{pagewarning}
	\setcounter{pagewarning}{\pagelimit}
	\AddEverypageHook{
		\ifnum\thepage>\thepagewarning
			\ifnum\LNCSpreview=0
				\begin{textblock}{1}[0.5,0](0,-13.5)
			\else
				\begin{textblock}{1}[0.5,0](0,-11.95)
			\fi
				\centering\Large
				\textcolor{red}{\textbf{This page is exceeding the page limit.}}
			\end{textblock}
		\fi
}
\fi

\def\notesindocument{0}

\newlength{\strutdepth}%
\settodepth{\strutdepth}{\strutbox}%

\newcommand{\notes}[3]{ %framework for margin notes of any kind
\ifnum\authnotes=1
	\def\notesindocument{1}%
	\noindent{%\bfseries
	\color{#1}{#3}\color{#1}}%
	\strut\vadjust{\kern-\strutdepth%
		\vtop to \strutdepth{%
			\baselineskip\strutdepth%
			\vss\llap{{\large\color{#1}\textbf{#2}\quad\color{black}}}\null%
		}%
	}%
\fi
}

% below you see how to set up margin notes. hspace hack for using paper size cropped to lncs size
\ifnum\LNCSpreview=0
	\newcommand{\todo}[1]{\notes{Red}{TODO}{#1}}
	\newcommand{\new}[1]{\notes{ForestGreen}{NEW}{#1}}
	\newcommand{\alert}[1]{\notes{Red}{ALERT}{#1}}
	\newcommand{\authornote}[1]{\notes{black}{NOTE}{#1}}
\else
	\newcommand{\todo}[1]{\notes{Red}{Todo\hspace{-1.5ex}}{#1}}
	\newcommand{\new}[1]{\notes{ForestGreen}{New\hspace{-1.5ex}}{#1}}
	\newcommand{\alert}[1]{\notes{Red}{Alert\hspace{-1.5ex}}{#1}}
	\newcommand{\authornote}[1]{\notes{black}{Note\hspace{-1.5ex}}{#1}}
\fi

\newcommand{\authnote}[2]{ %command for authnotes that appear in the text body
	\ifnum\authnotes=1
		\def\notesindocument{1}
		\begin{center}
			\fbox{%
				\begin{minipage}{.98\textwidth}
					\textbf{#1 says:} #2\authornote{}
				\end{minipage}%
			}
		\end{center}
	\fi
}

\ifnum\llncs=1
	\ifnum\llncsqedsymbol=1
		\let\oldproof\proof
		\renewenvironment{proof}%
		{\begin{oldproof}}%
		{\qed\end{oldproof}}
	\fi
\fi

%
\ifnum\llncs=0				%definition of new environemnt styles when not using llncs
\newtheoremstyle{mytheorem}
  {\topsep} % Space above
  {\topsep} % Space below
  {\itshape} % Body font
  {} % Indent amount
  {\bfseries} % Theorem head font
  {} % Punctuation after theorem head
  {.5em} % Space after theorem head
  {\thmname{#1}\thmnumber{ #2}\thmnote{ {\bfseries (#3).}}} % Theorem head spec

\newtheoremstyle{myplain}
  {\topsep} % Space above
  {\topsep} % Space below
  {\itshape} % Body font
  {} % Indent amount
  {\bfseries} % Theorem head font
  {} % Punctuation after theorem head
  {.5em} % Space after theorem head
  {\thmname{#1}\thmnumber{ #2}\thmnote{ {\normalfont(#3).}}} % Theorem head spec

\newtheoremstyle{mydefinition}
  {} % Space above
  {} % Space below
  {} % Body font
  {} % Indent amount
  {\bfseries} % Theorem head font
  {} % Punctuation after theorem head
  {.5em} % Space after theorem head
  {\thmname{#1}\thmnumber{ #2}\thmnote{ {\normalfont(#3).}}} % Theorem head spec

\newtheoremstyle{myremark}
  {} % Space above
  {} % Space below
  {} % Body font
  {} % Indent amount
  {\itshape} % Theorem head font
  {} % Punctuation after theorem head
  {.5em} % Space after theorem head
  {\thmname{#1}\thmnumber{ #2}\thmnote{ {\normalfont(#3).}}} % Theorem head spec

	\theoremstyle{mytheorem}				% new theorem-like environments when llncs=0
	\newtheorem{theorem}{Theorem}[section]

	\theoremstyle{myplain}
	\newtheorem{lemma}[theorem]{Lemma}
	\newtheorem{corollary}[theorem]{Corollary}
	\newtheorem{proposition}[theorem]{Proposition}
	\newtheorem{construction}[theorem]{Construction}
	\newtheorem{conjecture}[theorem]{Conjecture}

	\theoremstyle{mydefinition}
	\newtheorem{definition}[theorem]{Definition}
	\newtheorem{claim}[theorem]{Claim}
	\newtheorem{assumption}[theorem]{Assumption}
	\newtheorem{fact}[theorem]{Fact}

	\theoremstyle{myremark}
	\newtheorem{remark}[theorem]{Remark}
%	\newtheorem{example}[theorem]{Example}
	\newtheorem{note}[theorem]{Note}
	\newtheorem{observation}[theorem]{Observation}
\else									% new theorem-like environments when llncs=1
	\spnewtheorem{construction}[theorem]{Construction}{\bfseries}{}
	\spnewtheorem{assumption}[theorem]{}{\bfseries}{}
	\spnewtheorem{fact}[theorem]{Fact}{\bfseries}{}

	\spnewtheorem{observation}[theorem]{Observation}{\itshape}{}
\fi

% to save some typing
\newenvironment{theo}{\begin{theorem}}{\end{theorem}}
\newenvironment{thm}{\begin{theorem}}{\end{theorem}}
\newenvironment{lemm}{\begin{lemma}}{\end{lemma}}
\newenvironment{lem}{\begin{lemma}}{\end{lemma}}
\newenvironment{coro}{\begin{corollary}}{\end{corollary}}
\newenvironment{cor}{\begin{corollary}}{\end{corollary}}
\newenvironment{prop}{\begin{proposition}}{\end{proposition}}
\newenvironment{conj}{\begin{conjecture}}{\end{conjecture}}
\newenvironment{clai}{\begin{claim}}{\end{claim}}
\newenvironment{defi}{\begin{definition}}{\end{definition}}
\newenvironment{defn}{\begin{definition}}{\end{definition}}
\newenvironment{assu}{\begin{assumption}}{\end{assumption}}
\newenvironment{rema}{\begin{remark}}{\end{remark}}
\newenvironment{rem}{\begin{remark}}{\end{remark}}
\newenvironment{cons}{\begin{construction}}{\end{construction}}
\newenvironment{obse}{\begin{observation}}{\end{observation}}
\newenvironment{fct}{\begin{fact}}{\end{fact}}


\ifnum\abbrevref=0			%Tell cleveref what words to use in front of a reference (abbrevref=0)
	\crefname{assumption}{Assumption}{Assumptions}
	\crefname{construction}{Construction}{Constructions}
	\crefname{corollary}{Corollary}{Corollaries}
	\crefname{conjecture}{Conjecture}{Conjectures}
	\crefname{definition}{Definition}{Definitions}
	\crefname{exmaple}{Example}{Examples}
	\crefname{lemma}{Lemma}{Lemmata}
	\crefname{observation}{Observation}{Observations}
	\crefname{proposition}{Proposition}{Propositions}
	\crefname{remark}{Remark}{Remarks}
	\crefname{theorem}{Theorem}{Theorems}
\else						%Tell cleveref what words to use in front of a reference (abbrevref=1)
	\crefname{assumption}{Ass.}{Ass.}
	\crefname{construction}{Constr.}{Constr.}
	\crefname{corollary}{Cor.}{Cor.}
	\crefname{conjecture}{Conj.}{Conj.}
	\crefname{definition}{Def.}{Def.}
	\crefname{exmaple}{Ex.}{Ex.}
	\crefname{lemma}{Lem.}{Lem.}
	\crefname{observation}{Obs.}{Obs.}
	\crefname{proposition}{Prop.}{Prop.}
	\crefname{remark}{Rem.}{Rem.}
	\crefname{theorem}{Thm.}{Thms.}
\fi

\crefname{claim}{Claim}{Claims}
\crefname{fact}{Fact}{Facts}
\crefname{note}{Note}{Notes}

%Super Mario coin version of \getsr
\def\YYYSMcoin{\mbox{\begin{tikzpicture}[scale=0.0125]
\definecolor{coinbrown}{HTML}{D89E36}\definecolor{coindarkyellow}{HTML}{F8D81E}\definecolor{coinyellow}{HTML}{F8F800}\fill[coinyellow] (3,-1) rectangle (9,9);\fill(0,0) rectangle (1,8);\fill(1,8) rectangle (2,10);\fill(2,10) rectangle (4,11);\fill(4,11) rectangle (8,12);\fill(8,11) rectangle (10,10);\fill(10,10) rectangle (11,8);\fill(11,8) rectangle (12,0);\fill(10,-2) rectangle (11,0);\fill(8,-3) rectangle (10,-2);\fill(4,-4) rectangle (8,-3);\fill(2,-3) rectangle (4,-2);\fill(1,0) rectangle (2,-2);\fill (5,-1) rectangle (7,0);\fill (7,0) rectangle (8,8);\fill[coinbrown] (9,8) rectangle (10,10);\fill[coinbrown] (10,0) rectangle (11,8);\fill[coinbrown] (9,-2) rectangle (10,0);\fill[coinbrown] (8,-2) rectangle (9,-1);\fill[coinbrown] (4,-3) rectangle (8,-2);\fill[coindarkyellow] (2,-2) rectangle (3,8);\fill[coindarkyellow] (3,-2) rectangle (8,-1);\fill[coindarkyellow] (8,-1) rectangle (9,0);\fill[coindarkyellow] (9,0) rectangle (10,8);\fill[coindarkyellow] (8,8) rectangle (9,10);\fill[coindarkyellow] (4,9) rectangle (8,10);\fill[coindarkyellow] (3,8) rectangle (4,9);\fill[coindarkyellow] (5,0) rectangle (7,2);\fill[coindarkyellow] (6,2) rectangle (7,8);\fill[white] (4,0) rectangle (5,8);\fill[white] (5,8) rectangle (7,9);
\end{tikzpicture}}}

%Die version of getsr
\def\YYYdie{\mbox{\begin{tikzpicture}[scale=0.85,x=1em,y=1em,radius=0.09]
\draw[rounded corners=1,line width=.25pt] (0,0) rectangle (1,1);\fill (0.275,0.275) circle;\fill (0.725,0.725) circle;\fill (0.5,0.5) circle;\fill (0.275,0.725) circle;\fill (0.725,0.275) circle;
\end{tikzpicture}}}

\newcommand{\getsr}{
	\ifnum\dieordollar=0
		\mathrel{\vbox{\offinterlineskip\ialign{
			\hfil##\hfil\cr
			\hspace{0.1em}$\scriptscriptstyle\$$\cr
			$\leftarrow$\cr
		}}}
	\fi
	\ifnum\dieordollar=1
		\mathrel{\vbox{\offinterlineskip\ialign{
			\hfil##\hfil\cr
			{\scalebox{0.5}{\hspace{0.4em}\YYYdie}}\cr
			\noalign{\kern0.05ex}
			$\leftarrow$\cr
		}}}
	\fi
	\ifnum\dieordollar=2
		\mathrel{\vbox{\offinterlineskip\ialign{
			\hfil##\hfil\cr
			\hspace{0.1em}$\YYYSMcoin$\cr
			$\leftarrow$\cr
		}}}
	\fi
}

\newcommand{\testequal}{
	\mathrel{\vbox{\offinterlineskip\ialign{
			\hfil##\hfil\cr
			\hspace{0.1em}$\scriptscriptstyle?$\cr
			$\leftarrow$\cr
		}}}
}
\newcommand{\checkfornotes}{
	\ifnum\notesindocument=1
		\ifnum\LNCSpreview=1
			\begin{textblock}{1}[0.5,0](0,0.25)
		\else
			\begin{textblock}{1}[0.5,0](0,0.85)
		\fi
		\centering
		\textcolor{red}{\large \textbf{There are still unresolved comments in this document.}}
		\end{textblock}
	\fi
}





\usepackage[
lambda, % or n
advantage,
operators,
sets,
adversary,
landau,
probability,
notions,
logic,
ff,
mm,
primitives,
events, 
complexity,
asymptotics,
%oracles,
keys]{cryptocode}
\usepackage[hyperpageref]{backref}
\usepackage{stmaryrd}

%%%%%%%%%%%%%%%%%%%%%%%%%%%%%%%%%%%%%%%%%%%%%%%%%%%%%%%%%%%%%%%%%%%%%%%%%%%%%%%%%%%%%%%%%%%
%
%     SOME COMMANDS FOR FONTS
%
%%%%%%%%%%%%%%%%%%%%%%%%%%%%%%%%%%%%%%%%%%%%%%%%%%%%%%%%%%%%%%%%%%%%%%%%%%%%%%%%%%%%%%%%%%%

\newcommand{\varfont}[1]{\mathsf{#1}}
\newcommand{\algfont}[1]{\mathsf{#1}}
\newcommand{\notionfont}[1]{\mathsf{#1}}
\newcommand{\termfont}[1]{\mathsf{#1}}
\newcommand{\orfont}[1]{{\ensuremath{\textsc{#1}}}\xspace}
\newcommand{\eventfont}[1]{\mathsf{#1}}
\newcommand{\mat}[1]{\vec{#1}}
\newcommand{\QNum}[1]{\varfont{Q}_{#1}}
\newcommand{\setfont}[1]{\mathcal{#1}}
\newcommand{\Tran}{{\!\scriptscriptstyle{\top}}}
\newcommand{\Orth}{{\!\scriptscriptstyle{\bot}}}
\newcommand{\Dist}{\mathcal{D}}
%Useful tools
\newcommand{\BEval}[1]{\llbracket#1\rrbracket}

\renewcommand{\pp}{\varfont{pp}} %public parameters
\newcommand{\ct}{\varfont{ct}} %ciphertext
\newcommand{\List}[1]{\mathcal{L}_{#1}}
\newcommand{\pccheck}{\textbf{check}~}

%formatting
\newcommand{\Prop}[1]{\noindent\underline{\textsf{#1}}}

%distributions
\renewcommand{\sample}{\hookleftarrow}
\newcommand{\Unif}[1]{\mathsf{U}(#1)}

%math symbols
\newcommand{\trans}{\intercal}


%%%%%%%%%%%%%%%%%%%%%%%%%%%%%%%%%%%%%%%%%%%%%%%%%%%%%%%%%%%%%%%%%%%
%       Known term names for functional encryption               
%%%%%%%%%%%%%%%%%%%%%%%%%%%%%%%%%%%%%%%%%%%%%%%%%%%%%%%%%%%%%%%%%%%
\newcommand{\FE}{\termfont{FE}}
\newcommand{\IPFE}{\termfont{IPFE}}

%special property
\newcommand{\FH}{\termfont{FH}} % function hiding
\newcommand{\PFH}{\termfont{PFH}} % partially function hiding

\newcommand{\PFHIPFE}{\PFH\text{-}\IPFE}


%%%%%%%%%%%%%%%%%%%%%%%%%%%%%%%%%%%%%%%%%%%%%%%%%%%%%%%%%%%%%%%%%%%
%       Specific for partially function-hiding               
%%%%%%%%%%%%%%%%%%%%%%%%%%%%%%%%%%%%%%%%%%%%%%%%%%%%%%%%%%%%%%%%%%%
\newcommand{\hidef}{\algfont{hf}}
\newcommand{\TagSpace}{\setfont{T}}
\newcommand{\Tag}{\varfont{t}}
\newcommand{\SimEnc}{\algfont{SimEnc}}
%Notions
\newcommand{\SIG}{\algfont{SIG}}

\newcommand{\MSig}{\algfont{MSIG}}
\newcommand{\KGen}{\algfont{KGen}}
\newcommand{\Sig}{\algfont{Sig}}
\newcommand{\Parties}{\mathcal{P}}
\newcommand{\Exec}{\algfont{Exec}}
\newcommand{\Agg}{\algfont{Agg}}
\newcommand{\VerAgg}{\algfont{VerAgg}}
\newcommand{\MsgRec}{\algfont{MsgRec}}

%variables
\newcommand{\msg}{\varfont{m}}
\newcommand{\pmsg}{\varfont{pm}}
\newcommand{\NumP}{\varfont{N}}
\newcommand{\MSet}{\mathcal{M}}
\newcommand{\PkSet}{\mathcal{P}}
\newcommand{\SkSet}{\mathcal{S}}

\newcommand{\InSet}{\setfont{X}}
\newcommand{\OutSet}{\setfont{Y}}
\newcommand{\SigDim}{\varfont{m}}

%security
\newcommand{\mseufcma}{\notionfont{MS}\text{-}\notionfont{EUF}\text{-}\notionfont{CMA}}
\newcommand{\SecGame}[2]{\notionfont{Game}_{#1}^{#2}}
\newcommand{\mseufcmaG}{\SecGame{\MSig, \adv}{\mseufcma}}
%algorithms

\providecommand{\Prove}{\algfont{Prove}}
\providecommand{\Verif}{\algfont{Verif}}

%variables
\providecommand{\statmnt}{\varfont{x}}
\providecommand{\wit}{\varfont{w}}
%algorithms for the vector commitment schemes
\newcommand{\VCOM}{\algfont{VCS}}
\newcommand{\PCOM}{\algfont{PCS}}

%specific variables for the vector commitment schemes
\newcommand{\MsgSp}{\setfont{M}}
\newcommand{\ComSp}{\setfont{C}}
\newcommand{\OpenSp}{\setfont{OP}}
\newcommand{\TagSp}{\setfont{T}}
\newcommand{\MsgDim}{\varfont{d}}
\newcommand{\open}{\varfont{open}}

%scheme names
\newcommand{\KZG}{\algfont{KZG}}
\newcommand{\FRI}{\algfont{FRI}}

\newcommand{\SamplePre}{\algfont{SamplePre}}

%algorithms
\newcommand{\PTrap}{\algfont{PTF}}
\newcommand{\LH}{\algfont{LH}\text{-}}


%variables
\newcommand{\FOrd}{\varfont{q}}
\newcommand{\Zq}{\mathbb{Z}_\FOrd}
\newcommand{\tk}{\varfont{tk}}
\newcommand{\calg}[1]{\underline{\textbf{Alg} #1:}}
\newcommand{\coracle}[1]{\underline{\textbf{Oracle} #1:}}
%variables
\newcommand{\PolyDeg}{\varfont{d}}
\newcommand{\VarNum}{\varfont{\ell}}
\newcommand{\PVar}{\varfont{X}}

\newcommand{\sdh}{\notionfont{SDH}}

%group setting
\newcommand{\Ggen}{\varfont{g}}
\newcommand{\GGen}{\algfont{GGen}}
\newcommand{\GGA}{\GG_1}
\newcommand{\GGB}{\GG_2}
\newcommand{\GGT}{\GG_\notionfont{T}}
\newcommand{\GA}[1]{\begin{bmatrix}
#1
\end{bmatrix}_1}
\newcommand{\GB}[1]{\begin{bmatrix}
    #1
\end{bmatrix}_2}
\newcommand{\GT}[1]{\begin{bmatrix}
    #1
\end{bmatrix}_\notionfont{T}}
\newcommand{\G}[1]{\begin{bmatrix}
    #1
    \end{bmatrix}}
\newcommand{\pdot}{\bullet}

\newcommand{\pairing}[1]{e(#1)}
%classical notions
\newcommand{\pke}{\notionfont{PKE}}
\newcommand{\ibe}{\notionfont{IBE}}
\newcommand{\we}{\notionfont{WE}}
\newcommand{\abe}{\notionfont{ABE}}
\newcommand{\GL}{\notionfont{GL}}


%security notions
\newcommand{\cpa}{\notionfont{CPA}}
\newcommand{\cca}{\notionfont{CCA}}
\newcommand{\Rel}{\mathcal{R}}
\newcommand{\Lang}{\mathcal{L}}
\newcommand{\hb}{\algfont{hb}}

%variables
\newcommand{\InVar}{\varfont{x}}
\newcommand{\OutVar}{\varfont{y}}

\newcommand{\UC}{\notionfont{UC}}

\newcommand{\ucproto}{\notionfont{protocol}}
\newcommand{\ucadv}{\notionfont{adversary}}
\newcommand{\ucenv}{\notionfont{env}}
\newcommand{\ucidealf}{\notionfont{idealf}}
\newcommand{\ucsim}{\notionfont{sim}}
\newcommand{\ucinput}{\notionfont{input}}
\newcommand{\ucsubOut}{\notionfont{subOut}}
\newcommand{\ucsub}{\notionfont{subroutine}}
\newcommand{\ucbackdoor}{\notionfont{backdoor}}
\newcommand{\uccaller}{\notionfont{caller}}
\newcommand{\ucsubsid}{\notionfont{subsidiary}}

\newcommand{\Protocol}{\varfont{\pi}}
\newcommand{\Machine}{\varfont{\mu}}
\newcommand{\ucF}{\mathcal{F}}
\newcommand{\ucAdv}{\mathcal{A}}
\newcommand{\ucSim}{\mathcal{S}}
\newcommand{\ucEnv}{\mathcal{E}}
\newcommand{\NumMachines}{\varfont{N}}
\newcommand{\ComSet}{\varfont{Com}}
\newcommand{\Program}{\varfont{\tilde{\mu}}}


%variables
\newcommand{\NumIn}{\varfont{m}}
\newcommand{\NumOut}{\varfont{n}}
\newcommand{\EXEC}{\varfont{EXEC}}
%Do not need that due to our own macros
%% PLEASE do not change any of the already defined commands in this file. However, feel free to redefine commands or fonts if you feel an urge to do so in files of your own.

% If you want to define a new command for every lower (resp. upper) case latin character, add it to the group of instructions in lines 6-10, (resp. 14-15). Note that cal, adv, frak, bb, bf (resp. frak, bf) are already defined for all lower (resp. upper) case characters.

\def\makeuppercase#1{
\expandafter\newcommand\csname cal#1\endcsname{\mathcal{#1}}
\expandafter\newcommand\csname adv#1\endcsname{\mathcal{#1}}
\expandafter\newcommand\csname frak#1\endcsname{\mathfrak{#1}}
\expandafter\newcommand\csname bb#1\endcsname{\mathbb{#1}}
\expandafter\newcommand\csname bf#1\endcsname{\textbf{#1}}
}

\def\makelowercase#1{
\expandafter\newcommand\csname frak#1\endcsname{\mathfrak{#1}}
\expandafter\newcommand\csname bf#1\endcsname{\textbf{#1}}
}

\newcounter{char}
\setcounter{char}{1}

\loop
	\edef\letter{\alph{char}}
	\edef\Letter{\Alph{char}}
	\expandafter\makelowercase\letter
	\expandafter\makeuppercase\Letter
	\stepcounter{char}
	\unless\ifnum\thechar>26
\repeat

%%%%%%%%%%%%%%%%%%%%%%%%%%%%%%%%%%%%%%%%%%%%%%%%%%%%%%%%%%%%%%%%%%%%%%%%%%%%%%%%%%%%%%%%%%%
%
%     PLEASE CONSIDER STICKING TO THE STYLE FORCED UPON YOU BY THE FOLLOWING COMMANDS
%
%%%%%%%%%%%%%%%%%%%%%%%%%%%%%%%%%%%%%%%%%%%%%%%%%%%%%%%%%%%%%%%%%%%%%%%%%%%%%%%%%%%%%%%%%%%

\DeclareMathAlphabet{\mathsc}{OT1}{cmr}{m}{sc}

%%% PUBLIC KEY ENCRYPTION %%%
\newcommand{\PKE}{\textsf{PKE}}

\newcommand{\PKEGen}{\mathsf{Gen}}
\newcommand{\PKEEnc}{\mathsf{Enc}}
\newcommand{\PKEDec}{\mathsf{Dec}}

\newcommand{\PKEGenO}{\mathsc{Gen}} % in case you want to distinguish between an algorithm (here: in mathsf) and an oracle providing access to it.
\newcommand{\PKEEncO}{\mathsc{Enc}}
\newcommand{\PKEDecO}{\mathsc{Dec}}

\newcommand{\pk}{\mathit{pk}}
\newcommand{\sk}{\mathit{sk}}


%%% GAMES %%%
\newcommand{\INDCPA}{\mathsf{IND\text{-}CPA}}
\newcommand{\INDCCA}{\mathsf{IND\text{-}CCA}}

%%% SETS, SAMPLING %%%
\newcommand\bits{\{0,1\}}
\newcommand\cross{\times}

\newcommand\N{\bbN}
\newcommand\Q{\bbQ}
\newcommand\R{\bbR}
\newcommand\Z{\bbZ}

\newcommand\Zp{\bbZ_p}
\newcommand\Zq{\bbZ_q}
\newcommand\Zn{\bbZ_n}
\newcommand\ZN{\bbZ_N}
\newcommand\Fp{\bbF_p}

%%% MISC %%%

\newcommand{\tuple}[1]{\langle{#1}\rangle}
\newcommand{\xor}{\oplus}

\newcommand{\ceil}[1]{\left\lceil #1 \right\rceil}
\newcommand{\floor}[1]{\left\lfloor #1 \right\rfloor}
\newcommand{\abs}[1]{\left| #1 \right|}

\newcommand{\pr}[2][]{\Pr_{#1}\mathopen{}\left[#2\right]\mathclose{}}


%%%%%%%%%%%%%%%%%%%%%%%%%%%%%%%%%%%%%%%%%%%%%%%%%%%%%%%%%%%%%%%%%%%%%%%%%%%%%%%%%%%%%%%%%%%
%
%     MISC
%
%%%%%%%%%%%%%%%%%%%%%%%%%%%%%%%%%%%%%%%%%%%%%%%%%%%%%%%%%%%%%%%%%%%%%%%%%%%%%%%%%%%%%%%%%%%

\newcommand{\heading}[1]{{\vspace{1ex}\noindent\sc{#1}}}
\newcommand{\comment}{/\!\!/\,}
\mathchardef\hyphen="2D
\newcommand{\etal}{et~al.\@}
% `:=' looks ugly in LaTeX. The following code replaces it by a symmetric version of :=
% Simply keep using := as before.
\mathchardef\ordinarycolon\mathcode`\:
\mathcode`\:=\string"8000
\begingroup \catcode`\:=\active
  \gdef:{\mathrel{\mathop\ordinarycolon}}
\endgroup


%%%%%%%%%%%%%%%%%%%%%%%%%%%%%%%%%%%%%%%%%%%%%%%%%%%%%%%%%%%%%%%%%%%%%%%%%%%%%%%%%%%%%%%%%%%
%
%     COMMANDS FOR SEQUENCES OF GAMES
%
%%%%%%%%%%%%%%%%%%%%%%%%%%%%%%%%%%%%%%%%%%%%%%%%%%%%%%%%%%%%%%%%%%%%%%%%%%%%%%%%%%%%%%%%%%%

\newcommand{\newsequenceofgames}[1]{
  \newcounter{#1}
  \setcounter{#1}{-1}
  \def\sequencename{#1}

  \ifx \GameID \undefined
    \newcommand{\GameID}{#1}
  \else
    \renewcommand{\GameID}{#1}
  \fi

  \ifx \PrevLabel \undefined
    \newcommand{\PrevLabel}{\GameID.NULL}
  \else
    \renewcommand{\PrevLabel}{\GameID.NULL}
  \fi

  \ifx \ThisLabel \undefined
    \newcommand{\ThisLabel}{\GameID.NULL}
  \else
    \renewcommand{\ThisLabel}{\GameID.NULL}
  \fi
}

\newcommand{\nextgame}[1]{
  \let\PrevLabel\ThisLabel
  \renewcommand{\ThisLabel}{\GameID.#1}
  \refstepcounter{\GameID}\label{\GameID.#1}
  \paragraph{Experiment~$\mathsf{Exp}_\arabic{\GameID}.$}
}

\newcommand{\nextgamewithref}[2]{
  \let\PrevLabel\ThisLabel
  \renewcommand{\ThisLabel}{\GameID.#1}
  \refstepcounter{\GameID}\label{\GameID.#1}
  \paragraph{Game~\hyperref[#2]{$\mathsf{G}_\arabic{\GameID}.$}}
}

\newcommand{\gameref}[1]{%
{\hyperref[\sequencename.#1]{\mathsf{G}_{\ref{\sequencename.#1}}}}
}

\newcommand{\gamerefA}[1]{%
{\hyperref[\sequencename.#1]{\mathsf{G}^{\advA}_{\ref{\sequencename.#1}}}}
}

\newcommand{\gamedelta}[2]{%
\left|\Pr\left[\gamerefA{#1}\Rightarrow 1\right]-\Pr\left[\gamerefA{#2}\Rightarrow 1\right]\right|
}

\newcommand{\gameequal}[2]{%
\Pr\left[\gamerefA{#1}\Rightarrow 1\right]=\Pr\left[\gamerefA{#2}\Rightarrow 1\right]
}


%%%%%%%%%%%%%%%%%%%%%%%%%%%%%%%%%%%%%%%%%%%%%%%%%%%%%%%%%%%%%%%%%%%%%%%%%%%%%%%%%%%%%%%%%%%
%
%     SOME COMMANDS FOR PSEUDOCODE
%
%%%%%%%%%%%%%%%%%%%%%%%%%%%%%%%%%%%%%%%%%%%%%%%%%%%%%%%%%%%%%%%%%%%%%%%%%%%%%%%%%%%%%%%%%%%

% most of the commands are still in nicodemus.sty. However, e.g. texstudio will not allow for autocompletion of commands in nicodemus.sty :-/

\newenvironment{nicodemus}[1][\thenicolinenr]{%  % parameter controls initial line number
\begin{enumerate}[
topsep=0ex,
label=\nicolinenrformat\PaddingUp*,
ref=\nicorefprefix\PaddingUp*,
align=right,
leftmargin=0em,
itemindent=!,
labelindent=0em,
labelwidth=\nicolinenrwidth,
labelsep=\nicolinenrsep,
listparindent=\parindent,
noitemsep,
]%
\setcounter{enumi}{#1}%
\addtocounter{enumi}{-1}%
}{%
\end{enumerate}%
\addtocounter{enumi}{1}%
\setcounter{nicolinenr}{\theenumi}%
}

\newcommand{\nicoplus}{%
  \addtolength{\itemindent}{1em}%
  \addtolength{\labelsep}{1em}%
}
\newcommand{\nicominus}{%
  \addtolength{\itemindent}{-1em}%
  \addtolength{\labelsep}{-1em}%
}


% ---------------------------------------------------------------------
\usepackage{lmodern}
\usepackage[T1]{fontenc}
\usepackage[utf8]{inputenc}
\usepackage[final]{microtype}
% --- -----------------------------------------------------------------
% --- Add your required packages here.
% --- -----------------------------------------------------------------
\usepackage{macros}
%\usepackage[lambda]{cryptocode}
\usepackage{pseudocodemacros}

% --- Your package
% --- Your package

% ---------------------------------------------------------------------
% -------------         DOCUMENT BEGINS HERE          -----------------
% ---------------------------------------------------------------------

\begin{document}
% Just in case somebody should ever dare to touch those lines: Here's an explanation for provideauthors. The other commands provide... are constructed analogously.
%Please study the code block in main.tex that defines createauthor, createinstitute, first.
%Again, createinstitute is defined similarly to createauthor. On every call of createauthor{}{}{}{} a counter (starting at 0) is increased by 1 and the author's information is stored as a new command that looks like #name, #institute, ... for # in the naturals.
%Back to these commands: Provideauthors is defined recursively and adds 1name, 2name, 3name, ... to the authors as long as there are authors. All the \csname\endcsname commands are used to tell LaTeX to interpret it's content as a string and then consider it as a command. Otherwise, LaTeX won't understand commands like \1name, ....

\newcount\authorcounter
\newcommand{\provideauthors}{%
		\ifnum\authorcounter<\theauthorcount%if there is at least one more author to come
			\csname\the\authorcounter name\endcsname% output the current authorname 
			\expandafter\ifx\csname\the\authorcounter thanks\endcsname\empty%if thanks are not empty for that author, 
			\else
				\thanks{\csname\the\authorcounter thanks\endcsname}% outputs thanks
			\fi%
			\inst{\csname\the\authorcounter institute\endcsname} % add institute number to author.
			\and %places a , between authors and an ", and" between the last two (only for lncs)
			\global\advance\authorcounter by 1 % increase the counter that keeps track of how many authors have been placed
			\provideauthors%run provideauthors again
		\else% i.e. the current author is the last one. (Same code as above, but no \and after placing the author
			\csname\the\authorcounter name\endcsname % output the current authorname
			\expandafter\ifx\csname\the\authorcounter thanks\endcsname\empty %if thanks are not empty for that author, 
			\else
				\thanks{\csname\the\authorcounter thanks\endcsname} % outputs thanks
			\fi%
			\inst{\csname\the\authorcounter institute\endcsname} % add institute number to author.
		\fi
}

\def\atleastoneauthorplaced{0}
\newcommand{\providerunning}{%
	\ifnum\authorcounter<\theauthorcount%
		\expandafter\ifx\csname\the\authorcounter running\endcsname\empty
		\else
			\ifnum\authorcounter>1
				\ifnum\atleastoneauthorplaced=1
					\and%
				\fi
			\fi
			\csname\the\authorcounter running\endcsname
			\def\atleastoneauthorplaced{1}
		\fi
		\global\advance\authorcounter by 1
		\providerunning%
	\else%
		\expandafter\ifx\csname\the\authorcounter running\endcsname\empty
		\else
			\ifnum\authorcounter>1
				\ifnum\atleastoneauthorplaced=1
					\and%
				\fi
			\fi
			\csname\the\authorcounter running\endcsname
		\fi
	\fi
}

\newcount\institutecounter

\newcommand{\provideinstitutes}{%
	\ifnum\institutecounter<\theinstitutecount%
		\ifnum\llncs=0
			$^{\csname\the\institutecounter number\endcsname}$
		\fi
		\csname\the\institutecounter instname\endcsname
		
		\email{
			\ifx\contactmail\empty
				\csname\the\institutecounter mail\endcsname
			\else
				\href{mailto:\contactmail}{\csname\the\institutecounter mail\endcsname}
			\fi
		}
		
		\and%
			\global\advance\institutecounter by 1
		\provideinstitutes%
	\else%
		\ifnum\llncs=0
			\ifcsname 1name\endcsname
				$^{\csname\the\institutecounter number\endcsname}$
			\fi
		\fi
		\csname\the\institutecounter instname\endcsname
		
		\email{
			\ifx\contactmail\empty
				\csname\the\institutecounter mail\endcsname
			\else
				\href{mailto:\contactmail}{\csname\the\institutecounter mail\endcsname}
			\fi
		}
	\fi
}

\title{
	\ifnum\stuffedtitlepage=1
		\ifnum\llncs=1
			\vspace*{-7ex}
		\else
		\vspace*{-3ex}
		\fi
		\textbf{\titletext}
		\ifnum\llncs=1
			\vspace*{-2ex}
		\else
			\vspace*{-1ex}
		\fi
	\else
		\textbf{\titletext}
	\fi
}
\ifnum\anonymous=1
	\author{}
\else
	\ifnum\llncs=0
		\newcommand{\inst}[1]{{
			\ifcsname 1name\endcsname
				$^{#1}$
			\fi
			}}
	\fi
	\ifcsname 1name\endcsname
		\author{
			\global\authorcounter 1
			\provideauthors
		}
	\fi
\fi

\ifnum\llncs=1
	\titlerunning{\runningtitle}
	\ifnum\anonymous=1
		\institute{}
		\authorrunning{}
	\else
		\ifcsname 1instname\endcsname{
			\institute{
				\global\institutecounter 1
				\provideinstitutes
			}
		\fi
		\ifcsname 1name\endcsname{
			\authorrunning{
				\global \authorcounter 1
				\providerunning
			}
		\fi
	\fi
\fi
\maketitle
\ifnum\stuffedtitlepage=1
	\ifnum\llncs=0
		\vspace{-4ex}
	\fi
\fi

\ifnum\llncs=0
	\ifnum\anonymous=0
		\newcommand{\email}[1]{
			\texttt{
				\ifx\contactmail\empty
					#1
				\else
					\href{mailto:\contactmail}{#1}
				\fi
			}
		}
		\newcommand{\and}{}
		\ifnum\stuffedtitlepage=1
			\ifnum\llncs=0
				\vspace{-2ex}
			\fi
		\fi
		\begin{small}
			\begin{center}
				\global \institutecounter 1
				\provideinstitutes
			\end{center}
		\end{small}
	\fi
\fi

\ifnum\stuffedtitlepage=1
	\ifnum\llncs=1
		\vspace*{-4ex}
	\else
		\vspace*{-2ex}
	\fi
\fi
\ifnum\llncs=1
\begin{abstract}
	\abstracttext
	\vspace{1ex}

	\textbf{Keywords\ifnum\llncs=1{.}\else{:}\fi} \keywords
\end{abstract}
\fi
\ifnum\stuffedtitlepage=1
	\ifnum\llncs=1
		\vspace*{-2ex}
	\fi
\fi

\ifnum\llncs=0
	\vspace{1ex}
\fi

\ifx\choosepubinfo\empty\else
	\ifnum\choosepubinfo=1
	\def\pubinfo{\small
			\noindent \copyright\ IACR 
			\ifx\pubinfoYEAR\empty \textcolor{red}{year missing}\else \pubinfoYEAR\fi.
			This article is a minor revision of the version published by Springer-Verlag in the proceedings of \ifx\pubinfoCONFERENCE\empty \textcolor{red}{conference missing}\else \pubinfoCONFERENCE\fi, eventually available at \url{http://link.springer.com}
	}
	\fi
	
	\ifnum\choosepubinfo=2
		\def\pubinfo{\small
			\noindent \copyright\ IACR 
			\ifx\pubinfoYEAR\empty \textcolor{red}{year missing}\else \pubinfoYEAR\fi.
			This article is the final version submitted by the author(s) to the IACR and to Springer-Verlag on
			\ifx\pubinfoSUBMISSIONDATE\empty \textcolor{red}{submission date missing}\else \pubinfoSUBMISSIONDATE\fi.
			The version published by Springer-Verlag is available at
			\ifx\pubinfoDOI\empty \textcolor{red}{DOI missing}\else \pubinfoDOI\fi.
		}
	\fi
	
	\ifnum\choosepubinfo=3
		\def\pubinfo{\small
			\noindent \copyright\ IACR
			\ifx\pubinfoYEAR\empty \textcolor{red}{year missing}\else \pubinfoYEAR\fi.
			This article is a minor revision of the version published by Springer-Verlag available at
			\ifx\pubinfoDOI\empty \textcolor{red}{DOI missing}\else \pubinfoDOI\fi.
		}
	\fi
	
	\ifnum\choosepubinfo=4
		\def\pubinfo{\small
			\noindent This article is based on an earlier article:
			\ifx\pubinfoBIBDATA\empty \textcolor{red}{bibliographic data missing}\else \pubinfoBIBDATA\fi,
			\copyright\ IACR
			\ifx\pubinfoYEAR\empty \textcolor{red}{year missing}\else \pubinfoYEAR\fi,
			\ifx\pubinfoDOI\empty \textcolor{red}{DOI missing}\else \pubinfoDOI\fi.
		}
	\fi
	
	\ifnum\choosepubinfo=5
			\def\pubinfo{\small
				\noindent \pubinfoindividual
			}
		\fi
\fi

\textblockorigin{0.5\paperwidth}{0.9\paperheight}
\setlength{\TPHorizModule}{\textwidth}

\newlength{\pubinfolength}
\ifx\choosepubinfo\empty\else
	\ifnum\choosepubinfo=0
	\else
		\settowidth{\pubinfolength}{\pubinfo}
		\begin{textblock}{1}[0.5,0](0,.25)
			 \ifnum\pubinfolength<\textwidth
				\centering
			\fi
			\pubinfo
		\end{textblock}
	\fi
\fi
\thispagestyle{empty}

% --- -----------------------------------------------------------------
% --- Add your content here
% --- -----------------------------------------------------------------
\chapter{Lattices}
\section{Lattice trapdoors}

\subsection{Trapdoor of third-generation fully homomorphic encryption}

This is the gadget trapdoors from bit decomposition and rebuilding techniques, firstly presented in~\cite{EC:MicPei12,C:GenSahWat13}. Note that the lattice trapdoor of third generation is actually an "All-But-One"/"Punctured" trapdoor function or a punctured trapdoor function.

\begin{definition}[Punctured Trapdoor] A $\sis$-based punctured trapdoor scheme consists of three $\ppt$ algorithms $(\KGen, \TrapGen, \Eval, \SampleD, \SamplePre)$ with the following syntax:
\begin{trivlist}
    \item $\KGen(\secparam) \to \ek:$
    \item $\TrapGen(\secparam, \Tag^\star) \to (\ek, \tk_{\Tag^\star}):$
    \item $\Eval(\ek, \Tag, x) \to y:$
    \item $\SampleD(\ek, \Tag) \to x:$
    \item $\SamplePre(\ek, \tk_{\Tag_\MsgDim}, \Tag, y) \to x \cup \bot :$
\end{trivlist}

We also require the following properties.

\begin{trivlist}
    \item 
\end{trivlist}
\end{definition}

We give the following lattice-based trapdoor construction~\cite{EC:MicPei12}.



Informally, the above trapdoor function has linear homomorphism on the right side. We can compute $\sum_{j \in \setfont{J}} \Eval(\ek, \Tag, x_j) k_j = \sum_{j} \Eval(\ek, \Tag, x_j\cdot k_j)$ with some smallness restriction on the values of $k_j$ and $|\setfont{J}|$.

\begin{definition}[Almost Liear-Homomorphism]
    
\end{definition}


\chapter{Foundation}
\section{Hardcore Bit}~\label{sec:hardcore-bit}

The hardcore bit is introduced by~\cite{STOC:GolLev89}.

\subsection{Definition}

\begin{definition}[Hardcore Bit]
    Let $f:\InSet \to \OutSet$ be a function. A $(t, \varepsilon)$-hardcore bit is a function $\hb: \InSet \to \bin$ such that for any $\ppt$ adversary $\adv$ with running time $t$, we have
    \begin{align*}
        \left|\condprob{\adv(f(\InVar)) = \hb(\InVar)}{\InVar \sample \Unif{\InSet}}\right| &\leq \varepsilon.
    \end{align*}        
\end{definition}
\chapter{Commitment}
\input{chapters/commitment/basic}
\section{Dual-Mode Trapdoor Commitment}

\subsection{Definition}

\begin{definition}[Dual-Mode Trapdoor Commitment]
    
\end{definition}
\section{Vector Commitment}

Unavoidable references
\begin{itemize}
    \item \cite{TCC:LibYun10}: Concise mercurial vector commitments and independent zero-knowledge sets with short proofs.
    \item \cite{PKC:CatFio13}: Vector commitments and their applications.
\end{itemize}

\subsection{Definitions}


\begin{definition}[Vector Commitment] A vector commitment scheme consists of four $\ppt$ algorithms $\VCOM = (\Setup, \Com, \Open, \Ver)$ with the following syntax:
\begin{trivlist}
    \item $\Setup(\secparam, 1^\MsgDim) \to (\ck, \vk):$ Takes as input the security parameter $\secpar$, and the dimension of the message vector $\MsgDim$ as input, and returns a commitment key $\ck$ and a verification key $\vk$.
    \item $\Com(\ck, \msg) \to (\com,\open):$ Takes the commitment key $\ck$, and a message $\msg$ as input, and returns a commitment $\com$ and an opening information $\open$.
    \item $\Open(\ck, \com, \open, i \in [\MsgDim]) \to \pi_i:$ Takes a commitment key $\ck$, a commitment $\com$, an opening information $\open$, a position $i$ as input, and returns a proof $\pi_i$.
    \item $\Ver(\vk, \com, i, \msg, \pi_i) \to \bin:$ Takes a verification key $\vk$, a commitment $\com$, a position $i$, a message $\msg$ and a proof $\pi_i$ as input, and returns $1$ if the proof is accepted or $0$ otherwise.
\end{trivlist}

Additionally, we require the following properties

\begin{trivlist}
    \item \Prop{Correctness:}
    \item \Prop{Updatability:}
    \item \Prop{Stateless Updatability:}
    \item \Prop{Differentially Updatability:}
    \item \Prop{Position Binding:}
    \item \Prop{Hiding:} \footnote{In many applications, the hiding property of the vector commitments is not necessary.}
\end{trivlist}
\end{definition}

\subsection{Constructions}

\subsubsection{Lattice-based Vector Commitments}

This is the construction proposed by~\cite{TCC:PeiPepSha21}. The vector commitment is secure under the $\sis$ assumption. We use the linearly homomorphic puncturable trapdoor function $(\PTrap)$ as an abstraction of the lattice gadget in~\cite{EC:MicPei12}.

\begin{figure}[ht!]%
	\centering%
	\nicoresetlinenr
	\fbox{%
		\begin{minipage}[t]{.45\textwidth}%
            \calg{$\Setup(\secparam, 1^\MsgDim)$}
			\begin{nicodemus}
				\item $\Tag_0, \dots, \Tag_\MsgDim \sample \Unif{\TagSp}$
				\item $(\ek, \tk_{\Tag_\MsgDim}) \getsr \PTrap.\TrapGen(\ek, \tk_{\Tag_\MsgDim})$
				\item $y_0, \ldots, y_{\MsgDim-1} \sample \Unif{\OutSet}$
				\item \pcfor $i \in [\MsgDim]: x_{i,i} = 0$
				\item \pcfor $i, j \in [\MsgDim] \land i \neq j$
				\item \tab $x_{i,j} \getsr \PTrap.\SamplePre(\ek, \tk_{\Tag_\MsgDim}, \Tag_i, y_j)$
				\item $\ck \gets (y_0, \ldots, y_{\MsgDim-1}), \{x_{i,j}\}_{i,j \in [\MsgDim]}$
				\item $\vk \gets (ek, (y_0, \ldots, y_{\MsgDim-1}))$
				\item \pcreturn $(\ck, \vk)$
			\end{nicodemus}

			
		\end{minipage}
		\qquad
		\begin{minipage}[t]{.45\textwidth}
			\calg{$\Com(\ck, \msg)$}
			\begin{nicodemus}
				\item  $\com \gets \sum_{j \in [\MsgDim]}y_j \msg_j$
				\item $\open \gets \msg$
				\item \pcreturn $(\com, \open)$ 
			\end{nicodemus}
		\medskip
			\calg{$\Open(\ck, \com, \open, i)$}
			\begin{nicodemus}
				\item $\pi_i \gets \sum_{j \in [\MsgDim]}x_{i,j} \cdot  \msg_{j}$
				\item \pcreturn $\pi_i$
			\end{nicodemus}
		\medskip
			\calg{$\Ver(\vk, \com, i, \msg, \pi_i)$}
			\begin{nicodemus}
				\item \pcreturn  $\BEval{\PTrap.\Eval(\ek, \Tag_i, \pi_i) + y_i\msg_i = \com}$
			\end{nicodemus}
		\end{minipage}
	}%
	\caption{$\sis$-based vector commitment scheme.}
	\label{const:lat-vc}
\end{figure}

\subsubsection{Gadget-based Homomorphic Vector Commitment}

Reference: \cite{C:CMNW24}

This is the variate of  Ajtai commitment scheme~\cite{STOC:Ajtai96} with~\cite{EC:MicPei12} trapdoor.

\begin{figure}[ht!]%
	\centering%
	\nicoresetlinenr
	\fbox{%
		\begin{minipage}[t]{.45\textwidth}
			\calg{$\Setup(\secparam, 1^\MsgDim)$}%
			\begin{nicodemus}
				\item $\mat{A} \sample \Unif{\Zq^{n \times \MsgDim\log \FOrd}}$
				\item $\ck \gets \mat{A}$
				\item \pcreturn $\ck$
			\end{nicodemus}
		\medskip
			\calg{$\Com(\ck, \msg \in \Zq^\MsgDim)$} 
			\begin{nicodemus}
				\item 
			\end{nicodemus}
		\end{minipage}
		\qquad
		\begin{minipage}[t]{.45\textwidth}
			\coracle{}
			\begin{nicodemus}
				\item 
			\end{nicodemus}
		\end{minipage}
	}%
	\caption{Gadget-based $\sis$ homomorphic commitment.}
	\label{simple-figure}
\end{figure}




\section{Polynomial Commitments}

References:
\begin{itemize}
    \item \cite{AC:KatZavGol10}: KZG polynomial commitment scheme
    \item \cite{ICALP:BBHR18}: FRI polynomial commitment
    \item \cite{JC:FenMogNgu24}
    \item \cite{PKC:Libert24b}: Simulation-Extractable Polynomial commitment scheme from AGM and ROM.
\end{itemize}

\subsection{Definitions}

\begin{definition}[Polynomial Commitment~\cite{AC:KatZavGol10}]
    A polynomial commitment scheme consists of four $\ppt$ algorithms $\PCOM = (\Setup, \Com, \Eval, \Ver)$ with the following syntax.
    \begin{trivlist}
        \item $\Setup(\secparam,1^\VarNum, 1^\PolyDeg) \to (\ck, \tk):$ Takes a security parameter $\secpar$, a number of variables $\VarNum$ and polynomial degree of $\PolyDeg$ as input, and returns a commitment key $\ck$ and (optionally) a trapdoor key $\tk$.
        \item $\Com(\ck, f) \to (\com, \open):$ Takes a commitment key $\ck$ and a polynomial $f$ as input, and returns a commitment $\com$ and an opening information $\open$.
        \item $\Eval(\ck, \com, \open, \Tag, \vec{x}, y) \to \pi \cup \{\bot\}:$ Takes a commitment key $\ck$, a commitment $\com$, an opening information $\open$, (optionally) a tag $\Tag$, an input vector $\vec{x}$ and an output $y$ and input, and returns a proof $\pi$ if $f(\vec{x}) = y$ or a symbol $\bot$ otherwise.
        \item $\Ver(\ck, \com, \Tag, \vec{x}, y, \pi) \to \bin:$ Takes a commitment key $\ck$, a commitment $\com$, a tag $\Tag$, an input vector $\vec{x}$, an output $y$, and a proof $\pi$ as input, and returns $1$ if the proof is accepted or $0$ otherwise.
    \end{trivlist}
    We require the following properties.
    \begin{trivlist}
        \item \Prop{Correctness:}
        \item \Prop{Succinctness:} A polynomial commitment is succinct, if the size of the proof $\pi$ and the commitment $\com$ grows at most logarithmically with the degree $\PolyDeg$ of the committed polynomials.
        \item \Prop{Binding:}
        \item \Prop{Knowledge Soundness:}
        \item \Prop{Hiding:}
        \item \Prop{Zero-Knowledge:}
        \item \Prop{Simulation Extractability:}
    \end{trivlist}
\end{definition}

\subsection{Constructions}

We give the different constructions of polynomial commitments.


\subsubsection{KZG Polynomial Commitment}

We give the $\KZG$~\cite{AC:KatZavGol10} commitment scheme construction. The security of $\PCOM_{\KZG}$ is based on the $q\text{-}\sdh$ assumption in the symmetric pairing setting.

\begin{figure}[ht!]%
	\centering%
	\nicoresetlinenr
	\fbox{%
		\begin{minipage}[t]{.45\textwidth}
			\calg{$\Setup(\secparam, 1^\PolyDeg)$}%
			\begin{nicodemus}
				\item $( \GG, \GGT, \Ggen, \pdot) \getsr \GGen(\secparam)$
				\item $\alpha \sample \Unif{\Zq}$
				\item $\ck \gets (\G{1}, \G{\alpha}, \ldots, \G{\alpha^\PolyDeg})$
				\item \pcreturn $\ck$
			\end{nicodemus}
		\medskip
			\calg{$\Com(\ck, f)$} 
			\begin{nicodemus}
                \item $\com \gets \G{f(\alpha)} \in \GG$
                \item $\open \gets f$
                \item \pcreturn $(\com, \open)$
			\end{nicodemus}
		\end{minipage}
		\qquad
		\begin{minipage}[t]{.45\textwidth}
            \calg{$\Eval(\ck, \com, \open, x, y)$}
            \begin{nicodemus}
                \item \pccheck $\com = \G{f(\alpha)}$
                \item $h(\PVar) \gets \frac{f(\PVar)-f(x)}{\PVar-x}$
                \item $\pi \gets \G{h(\PVar)}$
                \item \pcreturn $\pi$
            \end{nicodemus}
        \medskip
			\calg{$\Ver(\ck, \com, x, y, \pi)$} 
			\begin{nicodemus}
				\item \pccheck $\com \pdot \G{1} = \pi \pdot \G{\alpha - x} + \GT{f(x)}$
			\end{nicodemus}
		\end{minipage}
	}%
	\caption{$\KZG$ polynomial commitment scheme.}
	\label{const:KZG-commit}
\end{figure}


\subsubsection{FRI Polynomial Commitment}

\subsubsection{}
\input{chapters/commitment/fcom.tex}
\chapter{Signature}
\section{Multi-Signatures}
\subsection{Litterature}
\begin{itemize}
    \item \cite{C:BosTakTib22}: DOTT: First Lattice-based One-Round Online multi-signature scheme
    \item \cite{C:BosTakTib22}: MuSig-L: Lattice-based multi-signature
    \item \cite{C:Chen23}: DualMS: lattice-based dual-mode multi-signature 
    \item \cite{EC:PanWag23}: First Multi-Signature without forking lemma
    \item \cite{PKC:PanWag22}: First Tightly-secure signature in the multi-user setting
    \begin{enumerate}
        \item Tight security in the multi-user setting requires random-self reducibility, which implies the tight worst-case to average-case reduction. Therefore, only LWE assumption is the potential candidate.
    \end{enumerate}
\end{itemize}

\subsection{Definition}

\begin{definition}[Multi-Signature]
    A (two-round) multi-signature scheme consists of four $\ppt$ algorithms $\MSig = (\Setup, \KGen, \Sig, \Ver)$ with the following syntax:
    \begin{itemize}
        \item $\Setup(\secparam) \to \pp$ takes the security parameter $\secparam$ as input and returns public parameter $\pp$. We assume that $\pp$ implicitly defines the space of public keys, secret keys, messages, and signatures, respectively. All algorithms related to $\MSig$ take at least implicitly $\pp$ as input.
        \item $\KGen(\pp) \to (\pk, \sk)$ takes the public parameter $\pp$ as input, and returns a public key $\pk$, and a secret key $\sk$.
        \item $\Sig = (\Sig_0, \Sig_1, \Sig_2)$ is a three staged algorithms:
        \begin{itemize}
            \item $\Sig_0(\Parties, \sk, \msg) \to (\pmsg_1, \st_1)$ takes a list of public keys $\Parties = (\pk_1, \ldots, \pk_{\NumP})$, a secret key $\sk$, and a message $\msg$ as input, and returns a protocol message $\pmsg_1$, and a state $\st_1$.
            \item $\Sig_1(\st_1, \MSet_1) \to (\pmsg_2, \st_2)$ takes as input a state $\st_1$ and a tuple of protocol messages messages $\MSet_1 = (\pmsg_{1,1}, \ldots, \pmsg_{1, \NumP})$, and returns a protocol message $\pmsg_2$ and a state $\st_2$.
            \item $\Sig_2(\st_2, \MSet_2) \to \sigma$ takes as input a state $\st_2$ and a tuple of protocol messages messages $\MSet_2 = (\pmsg_{2,1}, \ldots, \pmsg_{2, \NumP})$, and returns a signature $\sigma$.
        \end{itemize}
        \item $\Ver(\Parties, \msg, \sigma) \to b$ takes a set of public keys $\Parties = (\pk_1, \ldots, \pk_{\NumP})$, a message $\msg$, and a signature $\sigma$ as input, and returns a bit $b \in \{0,1\}$.
    \end{itemize}
    We require the following properties:
    \begin{itemize}
        \item \heading{Completeness:} $\MSig$ is complete if for all $\pp \getsr \Setup(\secparam)$, all $\NumP \in \poly$, all $(\pk_j, sk_j)\getsr \KGen(\pp)$ for $j \in [\NumP]$, all $\PkSet = (\pk_1, \ldots, \pk_\NumP)$, all $\SkSet = (\sk_1, \ldots, \sk_\NumP)$, and all messages $\msg$, we have
        \begin{align*}
            \condprob{\Ver(\PkSet, \msg, \sigma) = 1}{\sigma \getsr \MSig.\Exec(\PkSet, \SkSet, \msg)} = 1,
        \end{align*}
        where the algorithm $\MSig.\Exec$ is defined in \cref{fig:def-complet-multi-sig}.
        \begin{figure}[h!]%
	   \centering%
	   \nicoresetlinenr
	   \fbox{%
		\begin{minipage}[t]{.45\textwidth}%
			\underline{\textbf{Alg} $\Exec(\PkSet, \SkSet, \msg):$}
			\begin{nicodemus}
				\item \pcparse $\PkSet  =: (\pk_1, \ldots, \pk_\NumP)$
                    \item \pcparse $\SkSet =:(\sk_1, \ldots, \sk_\NumP)$
                    \item \pcfor $i \in [\NumP] : (\pmsg_{1,i}, \st_{1,i}) \getsr \Sig_0(\PkSet, \sk, \msg)$
                    \item $\MSet_1 \gets (\pmsg_{1,1}, \ldots, \pmsg_{1, \NumP})$
                    \item \pcfor $i \in [\NumP] : (\pmsg_{2,i}, \st_{2,i}) \getsr \Sig_1(\st_{1,i}, \MSet_1)$
                    \item $\MSet_2 \gets (\pmsg_{2,1}, \ldots, \pmsg_{2, \NumP})$
                    \item \pcfor $i \in [\NumP] : \sigma_i \getsr \Sig_2(\st_{1,i}, \MSet_2)$
                    \item \pcif $\exists i \neq j \in [\NumP]. \sigma_i \neq \sigma_j:$ \pcreturn $\bot$
                    \item \pcreturn $\sigma \gets \sigma_1$
			\end{nicodemus}
		\end{minipage}
	   }%
	   \caption{The execution algorithm $\Exec$ for a (two-round) multi-signature scheme $\MSig = (\Setup, \KGen, \Sig, \Ver)$, representing an honest execution of the signing protocol $\MSig$.}
	   \label{fig:def-complet-multi-sig}
        \end{figure}
        \item \heading{Key Aggregation:} A multi-signature scheme $\MSig = (\Setup, \KGen, \Sig, \Ver)$ is said to support key aggregation, if the algorithm $\Ver$ can be split into two deterministic polynomial time algorithms $(\Agg, \VerAgg)$ with the following syntax:
        \begin{itemize}
            \item $\Agg(\PkSet) \to \bar{\pk}$ takes a public key set $\PkSet = (\pk_1, \ldots, \pk_\NumP)$ as input, and returns an aggregated public key $\bar{\pk}$.
            \item $\VerAgg(\bar{\pk}, \msg, \sigma) \to b$ is deterministic, takes an aggregated key $\bar{\pk}$, a message $\msg$, and a signature $\sigma$ as input, and returns a bit $b \in \{0,1\}$.
        \end{itemize}
        More precisely, algorithm $\Ver(\PkSet, \msg, \sigma)$ can be written as $\VerAgg(\Agg(\PkSet), \msg, \sigma)$. Note that we require succinctness of aggregated public key. Namely, the size of $\bar{\pk}$ is independent of the number of users.
        
        \item \heading{$\mseufcma$:} Let us consider the security game $\mseufcmaG$ defined in~\cref{fig:def-mseufcma}. We say that $\MSig$ is $\mseufcma$ secure, if for all $\ppt$ adversary $\adv$, the following advantage is negligible:
        \begin{align*}
            \mathsf{Adv}_{\MSig, \adv}^{\mseufcma}(\secparam) & \gets \prob{\mseufcmaG(\secparam) \Rightarrow 1}.
        \end{align*}

        \begin{figure}[ht!]%
	\centering%
	\nicoresetlinenr
	\fbox{%
		\begin{minipage}[t]{.45\textwidth}%
			$\underline{\mseufcmaG(\secparam)}$
			\begin{nicodemus}
				\item $\pp \getsr \Setup(\secparam)$ 0
                    \item $(\pk, \sk) \getsr \KGen(\pp)$
                    \item $\Oracle{\Sig} = (\Oracle{\Sig}_0, \Oracle{\Sig}_1, \Oracle{\Sig}_2)$
                    \item $(\PkSet^\star, \msg^\star, \sigma^\star) \getsr \adv^{\Oracle{\Sig}}(\pp, \pk)$
                    \item \pcif $\pk \notin \PkSet^\star \lor (\PkSet^\star, \msg^\star) \in \List{}:$
                    \item \tab \pcreturn $0$
                    \item \pcreturn $\Ver(\PkSet^\star, \msg^\star, \sigma^\star)$
			\end{nicodemus}
		\medskip
			\noindent \underline{\textbf{Oracle} $\Oracle{\Sig}_0(\PkSet, \msg):$}
			\begin{nicodemus}
				\item \pcparse $(\pk_1, \ldots, \pk_\NumP) =: \PkSet$
                    \item \pccheck $\pk_1 = \pk$
                    \item $\List{} \gets \List{} \cup \{(\PkSet,\msg)\}$
                    \item $sid \gets sid+1;\; ctr[sid] \gets 1$
                    \item $(\pmsg_1, \st_1)\getsr \Sig_0(\PkSet, \sk, \msg)$
                    \item $(\pmsg_1[sid], \st_1[sid]) \gets (\pmsg_1, \st_1)$
                    \item \pcreturn $(\pmsg_1[sid], sid)$
			\end{nicodemus}
		\end{minipage}
		\qquad
		\begin{minipage}[t]{.45\textwidth}
			\underline{\textbf{Oracle} $\Oracle{\Sig}_1(sid, \MSet_1):$}
			\begin{nicodemus}
				\item \pccheck $ctr[sid]\neq 1$
                    \item \pcparse $(\pmsg_{1,1}, \ldots, \pmsg_{1, \NumP}) =:\MSet_1$
                    \item \pccheck $\pmsg_1[sid] =  \pmsg_{1,1}$
                    \item $ctr[sid] \gets ctr[sid]+1$
                    \item $(\pmsg_2, \st_2) \getsr \Sig_1(\st_1[sid], \MSet_1)$
                    \item $(\pmsg_2[sid], \st_2[sid]) \gets (\pmsg_2, \st_2)$
                    \item \pcreturn $\pmsg_2[sid]$
			\end{nicodemus}
                \medskip
                \underline{\textbf{Oracle} $\Oracle{\Sig}_2(sid, \MSet_2)$}
                \begin{nicodemus}
				\item \pccheck $ctr[sid] = 2$
                    \item \pcparse $(\pmsg_{2,1}, \ldots, \pmsg_{2, \NumP}) =:\MSet_2$
                    \item \pccheck $\pmsg_2[sid] =  \pmsg_{2,1}$
                    \item $ctr[sid] \gets ctr[sid]+1$
                    \item $\sigma \getsr \Sig_2(\st_2[sid], \MSet_2)$
                    \item \pcreturn $\sigma$
			\end{nicodemus}
		\end{minipage}
	}%
	\caption{The security game $\mseufcmaG$ for a (two-round) multi-siganture scheme $\MSig$ and an adversary $\adv$. Without losing the generality, we assume the adversary cannot access $\sk$ corresponding to $\pk$, the first public key $\pk_1$ of the set $\PkSet$.}
	\label{fig:def-mseufcma}
        \end{figure}
    \end{itemize}
\end{definition}

\subsection{Some New Ideas}

Here we list some of the ideas:

\subsubsection{Tightly-secure Multi-Signature from $\cdh$ assumption}
We give some insights here
\begin{itemize}
    \item Our starting point is~\cite{C:Chevallier-Mames05}, which is a tightly-secure signature scheme based on $\cdh$ assumption.
    \item The problem we encounter here is that, as with all other multi-signature schemes, it is hard to simulate the signature without knowing the secret key. The core reason is that the adversary may generate its first-round message  $\mathsf{R}_1$ after seeing all honest ones. The potential solution is to use the random oracle as in~\cite{CCS:BelNev06}. Compare to~\cite{EC:PanWag23}, the random oracle is used as an equivocable commitment scheme.
    \item We do not need the homomorphism as in~\cite{EC:PanWag23}, this is because we directly open the commitment at round $2$. 
\end{itemize}

This does not work since we can only have $3$-round $\cdh$ based protocol. We need $2$ rounds only to check that the first commitments are correct in the ROM model. \cite{EC:PanWag23} does not need this because they homomorphically compute the commitment and check after the homomorphism. Maybe $3$-round tightly secure multi-signature from $\cdh$ assumption is feasible. But not sure this is a valuable result.

\subsubsection{Simpler generic construction of multi-signature scheme}

\begin{itemize}
    \item We can notice that the main difficulty of the construction of 
\end{itemize}
\chapter{Non-Interactive Zero-Knowledge Proof}
\section{Definition of Non-Interactive Zero-Knowledge Proof}

\begin{definition}
    A non-interactive zero-knowledge proof system consists of three $\ppt$ algorithms $\nizk = (\Setup, \Prove, \Verif)$ with the following syntax.
    \begin{itemize}
        \item $\Setup(\secparam) \to \crs:$ Takes a security parameter $\secpar$
        \item $\Prove(\crs, \statmnt, \wit) \to  \pi:$
        \item $\Verif(\crs, \statmnt, \pi) \to \bin:$
    \end{itemize}
\end{definition}



\section{Groth-Sahai proof system}

\subsection{References}
    \begin{itemize}
        \item 
    \end{itemize}
\chapter{Encryption schemes}
\section{Chosen-Ciphertext Encryption}

We try to give a list of chosen-ciphertext encryption constructions as exhaustive as possible.


There are two types of generic transform, with or without random oracle model. The transform without random oracle are more desirable, but more complex at the same time.

In the random oracle model, we give the following transforms.
\begin{itemize}
    \item Fujisaki-Okamoto~\cite{JC:FujOka13} transform
\end{itemize}

In the standard model, there are several different way to construct $\cca$-secure encryption scheme.
\begin{itemize}
    \item BCHK transform~\cite{SIAM:BCHK07}: this transform a selective-ID $\ibe$ into an $\cca$-secure encryption scheme.
    \item Lossy Trapdoor functions~\cite{STOC:PeiWat08}
    \item Naor-Yung transform~\cite{STOC:NaoYun90} 
\end{itemize}
\section{Puncturable Encryption}

The puncturable encryption scheme is firstly introduced in the~\cite{SP:GreMie15} to construction dynamic searchable encryption scheme.

\heading{Related Works:}
\begin{itemize}
    \item Functional Encryption scheme: the puncturable encryption scheme can be seen as a particular case of 
\end{itemize}

\subsection{Definition}


\section{Witness Encryption}

There is a strong connection between $\we$ and computational secret sharing for $\NP$~\cite{STOC:GGSW13}.

\heading{Related Works}
\begin{itemize}
    \item \cite{STOC:GGSW13} They have introduced the first witness encryption scheme.
    \begin{itemize}
        \item First definition of witness encryption.
        \item Construction of $\pke$, $\ibe$, $\abe$, adaptively-secure $\ibe$ schemes from $\we$.
        \item Construction of $\we$ from the multi-linear map.
    \end{itemize}
\end{itemize}

\subsection{Definition}

We define the witness encryption scheme here. It was firstly proposed in~\cite{STOC:GGSW13} as a concept for building attribute-based encryption schemes ($\abe$).

\begin{definition}
    A witness encryption scheme for an $\NP$ relation $\Rel \subseteq \bin^{n(\secpar)} \times \bin^{m(\secpar)}$ with a corresponding $\NP$ language $\Lang = \{\statmnt : \exists \wit. (\statmnt,\wit) \in \Rel\}$ consists of three $\ppt$ algorithms $\we = (\KGen, \Enc, \Dec)$ with the following syntax.
    \begin{trivlist}
        \item $\KGen(\secparam, \Rel) \to \ek:$
        \item $\Enc(\ek, \statmnt, \msg) \to \ct:$
        \item $\Dec(\ek, \ct, \wit) \to \msg:$
    \end{trivlist}
    We require the following properties:
    \begin{trivlist}
        \item \Prop{Correctness:} For all $\secpar \in \NN$, all $(\statmnt, \wit) \in \Rel$, and all $\msg \in \MsgSp$, we have:
        \begin{align*}
            \condprob{\Dec(\ek, \Enc(\ek, \statmnt, \msg)) = \msg}{\ek \getsr \KGen(\secparam, \Rel)} &= 1- \negl.
        \end{align*}
        \item \Prop{Security:} For all two-stages $\ppt$ adversary $\adv = (\adv_1, \adv_2)$, we have:
        \begin{align*}
            \left|\condprob{b' = b \land \statmnt \notin \Lang}{\begin{array}{l}
                \ek \getsr \KGen(\secparam, \Rel)\\
                (\msg_0, \msg_1, \statmnt, \state) \getsr \adv_1(\ek)\\
                b \getsr \bin\\
                \ct_b \getsr \Enc(\ek, \statmnt, \msg_b)\\
                b' \getsr \adv_2(\state, \ct_b)
            \end{array}}- \frac{1}{2}\right| & = \negl.
        \end{align*}
    \end{trivlist}
\end{definition}

\subsection{Applications}

\subsubsection{$\pke$ from $\we$}

\subsubsection{$\ibe$ from $\we$}

We present here the $\ibe$ construction from $\we$ schemes. We also need an unique signature scheme $\SIG = (\KGen, \Sig, \Ver)$ together with a Goldreich-Levin~\cite{STOC:GolLev89} hardcore bit $\hb$ (\cref{sec:hardcore-bit}).

We can introduce the notion of message-recoverable signature scheme, which means given a signature $\sigma$, it is easy to get a message $\msg$, such that $\SIG.\Ver(\vk, \msg, \sigma) =1$. Note that such scheme can be easily constructed from any signature scheme by appending the message directly after the signature. 

Note that given an unique signature $\sigma$ with message recoverability, the message recover algorithm $\SIG.\MsgRec(\mpk, \sigma) \to \hash(\msg)$ of signature scheme is actually an one-way function. Therefore, we can define the hardcore bit function $\hb$ of the above one-way function as $\hb_{\vec{r}}(\sigma)\gets \vec{r} \cdot \sigma$.

For the witness encryption scheme we define the following two $\NP$ relations.
\begin{align*}
    ((\id, \vec{r}), \sigma) \in \Rel_0 & \Leftrightarrow \SIG.\Ver(\mpk, \id,  \sigma) = 1 \land \hb_{\vec{r}}(\sigma) = 0 \\
    ((\id, \vec{r}), \sigma) \in \Rel_1 & \Leftrightarrow \SIG.\Ver(\mpk, \id,  \sigma) = 1 \land \hb_{\vec{r}}(\sigma) = 1
\end{align*}

\begin{figure}[ht!]%
	\centering%
	\nicoresetlinenr
	\fbox{%
		\begin{minipage}[t]{.45\textwidth}
			\calg{$\Setup(\secparam)$}%
			\begin{nicodemus}
				\item $(\vk, \sk) \getsr \SIG.\KGen(\secparam)$
				\item $\mpk \gets \vk;\; \msk \gets \sk$
				\item \pcreturn $(\mpk, \msk)$
			\end{nicodemus}
		\medskip
            \calg{$\Enc(\mpk, \id, \msg)$}
            \begin{nicodemus}
                \item $\SigDim \gets \log_2|\SIG.\Sigma|$
                \item $\vec{r} \gets \bin^\SigDim$
                \item $\pk_0 \getsr \we.\KGen(\secparam, \Rel_0)$
                \item $\pk_1 \getsr \we.\KGen(\secparam, \Rel_1)$
                \item $\ct_0 \getsr \we.\Enc(\pk_0, (\id, \vec{r}), \msg)$
                \item $\ct_1 \getsr \we.\Enc(\pk_1, (\id, \vec{r}), \msg)$
                \item \pcreturn $\ct = (\id, \vec{r}, \ct_0, \ct_1)$
            \end{nicodemus}
		\end{minipage}
		\qquad
		\begin{minipage}[t]{.45\textwidth}
			\calg{$\KGen(\msk, \id)$} 
			\begin{nicodemus}
				\item $\sigma \getsr \SIG.\Sig(\sk, \id)$
			\end{nicodemus}
        \medskip
            \calg{$\Dec(\sk_\id, \ct)$}
            \begin{nicodemus}
                \item \pcparse $(\id, \vec{r}, \ct_0, \ct_1) =: \ct$
                \item \pcif $\sigma^\trans \cdot \vec{r} = 0$ \pcthen
                \item \tab $\msg \gets \we.\Dec(\pk_0, \sigma, \ct_0)$
                \item \pcelse
                \item \tab $\msg \gets \we.\Dec(\pk_1, \sigma, \ct_1)$
                \item \pcreturn $\msg$
            \end{nicodemus}
		\end{minipage}
	}%
	\caption{Construction of $\ibe$ scheme from $\we$.}
	\label{const:we2ibe}
\end{figure}

\Prop{Proof intuitions:} We try to modify the challenge ciphertext such that the $\ct_0$ encrypts $\msg_b$ and $\ct_1$ encrypts $\msg_{1-b}$. If the adversary answers $0$, we will answer that the hardcore bit is $0$, we will answer $1$ otherwise. Note that since our objective if to have the winning probability is non-negligibly far from $\frac{1}{2}$, it does not matter if we guess $b$ to $1-b$.

{\color{red} In the original paper of~\cite{STOC:GGSW13}, the proof is actually wrong here, without knowing the secret signing key, the simulator cannot really determine whether we should change $\ct_0$ or $\ct_1$. On the other hand, if the simulator knows the signing key then the function from $\sigma$ to $\msg$ is no longer an one-way function.}

\subsubsection{$\abe$ from $\we$}

\subsubsection{Adaptively-secure $\ibe$ from $\we$}
\chapter{Open Problems}


\section{Some Ideas}

\begin{itemize}
    \item Use new lattice encoding to improve key exchange protocols' efficiency.
    \item Distributed Broadcast Encryption with Ring Signature
    \item MPC YOSO model theoretical lower bound
    \item DSSE TYPE-2
    \item Lattice signature tightness~\cite{PKC:DGJL21} explains why this is difficult in lattice setting. (Extremely Difficult)
    \begin{itemize}
        \item The main technical difficulty of tightly secure lattice scheme is basically the RSR only presents for LWE and false for RLWE or MLWE.
        \item This is related to the samples preserving lattice reductions.
    \end{itemize}
    \item More efficient adaptively-secure puncturable pseudorandom functions
    \begin{itemize}
        \item \cite{EC:Yang23} gives a construction of adaptively-secure private puncturable pseudorandom function.
    \end{itemize}
    \item Adaptively secure puncturable encryption
    \item Construction of more efficient vector commitment
\end{itemize}

% --- -----------------------------------------------------------------
% --- Put your bibliography files here
% --- -----------------------------------------------------------------

\bibliography{%	Please make sure you have no blanks left in here!
%	first file,second file, ... %
	cryptobib/abbrev3,cryptobib/crypto% Uncomment this line after adding the cryptobib submodule to use cryptobib.
	,add
}

% ------------- Please do not touch this block of code ----------------
\checkfornotes
\ifnum\anonymous=0
	\ifnum\acknowledgments=1
		\paragraph{Acknowledgments}
		\acknowledgmenttext
	\fi
\fi
\ifnum\llncs=1
	\bibliographystyle{splncs03}
\else
	\bibliographystyle{\choosebibstyle}
\fi
\appendix

% --- -----------------------------------------------------------------
% --- The Appendix starts here
% --- -----------------------------------------------------------------

%\section{Appendix}

% --- -----------------------------------------------------------------
% --- Add your content to be appended here
% --- -----------------------------------------------------------------

% If you use cleveref (as you do by default) and are required to rename
% the appendix (e.g. 'supplementary material'), use
% \crefname{appendix}{Supplementary Material}{Supplementary Material}
% syntax is \crefname{appendix}{new name singular}{new name plural}

\end{document}

% ---------------------------------------------------------------------
% ---------------          DOCUMENT ENDS HERE           ---------------
% ---------------------------------------------------------------------

% --- -----------------------------------------------------------------
% --- Almost completely rewritten template with vast improvements
% --- compared to the old version. Please feel free to contact me
% --- with feedback or questions felix.heuer@rub.de.
% --- -----------------------------------------------------------------